

This paper intends to show that the notion of authenticity as it relates
to heritage in Martin Heidegger's \emph{Being and Time}\footnote{Martin
  Heidegger, \emph{Being and Time}, trans. Joan Stambaugh (Albany: SUNY
  Press, 2010).} and later adopted by Hubert Dreyfus excludes genuinely
new ways of being. For both thinkers, being authentic qua world-changer
is a matter of retrieving marginal practices within one's own heritage.
I will firstly provide a foundation for the relevant area of Heidegger's
philosophy, then move on to explain how authenticity appears in
Dreyfus's interpretation of Heidegger. I will then introduce Dreyfus's
commentary on Jonathan Lear's \emph{Radical Hope} which will be used to
seek insight into his concept of the world-changer through his use of
concrete examples. Ultimately, the insight gleaned will be limited and
we will see the vague nature of Dreyfus's world-changer concept. Using
the thin information that can be deduced about the world-changer, I
provide three of my own examples that demonstrate the view's limitations
beyond its vagueness. In these examples---concerning a revolutionary
musical composer, one of the first women to undergo gender confirmation
surgery, and a person estranged from their heritage---we will see that
the Heideggerian view appropriated by Dreyfus excludes many new ways of
being, barring revolutionary actors from authenticity, while also
suffering a tension between the Heideggerian call of conscience and
retrieval. These examples will also reveal unjustified valuations
implicit to the critiqued view: the privileging of tradition over the
progressive and the use of hierarchical language. To conclude the paper,
I show how the exclusionary nature of authenticity follows from the
phenomenological method of \emph{Being and Time} and offer a preliminary
sketch on how actors could find authenticity independently of any
heritage.

Authenticity in the Heideggerian sense is a mode of being of Dasein. To
understand what this could mean, let us first consider the general
notion of Dasein: Heidegger's technical term for the distinct human way
of being. Heidegger sees that humans are the kind of beings that ask
about what their lives mean and have some understanding of the world
they inhabit and continuously construct through shared
practices.\footnote{Heidegger, \emph{Being and Time}, 12-13.} Heidegger
tells us that there are two modes of being by which we can understand
Dasein and thus the world: authenticity and inauthenticity.\footnote{\emph{Id}.,
  42-43.} Regarding Dasein that is inauthentic, Heidegger writes, ``It
understands itself in terms of the possibilities of existence that
`circulate' in the present day `average' public interpretedness of
Dasein.''\footnote{\emph{Id}., 383.} The inauthentic mode of being is to
engage with the world and its tools in the perfunctory manner of the
general public. Inauthentic Dasein only understands its possibilities of
action as the ordinary responses to life's situations. Dasein in this
mode of being takes its surroundings for granted, unaware of the fragile
infrastructure that gives our performances meaning. In contrast,
authentic Dasein responds to ``the call of conscience {[}which{]}
reveals the lostness in the they\ldots{} One's own potentiality-of-being
becomes authentic and transparent in the understanding
being-toward-death as one's \emph{ownmost} possibility.''\footnote{\emph{Id}.,
  307.} The authentic Dasein hears a call of conscience from itself
which directs the individual Dasein to see itself as an individual. The
call of conscience makes a demand on the individual Dasein that it must
take decisive action to take over its own life, and not to live
inattentively as the public does. The contents of that demand differ for
each Dasein, but only through following its demands can Dasein become
authentic. Dreyfus investigates Heidegger's authenticity and expands on
the kind of character that results from responding to the call of
conscience.

In Dreyfus's ``Could anything be more Intelligible than Everyday
Intelligibility?: Reinterpreting Division I of Being and Time in the
light of Division II'' (``Reinterpreting''),\footnote{Hubert L. Dreyfus,
  ``Could anything be more intelligible than everyday
  intelligibility? Reinterpreting division I of \emph{Being and Time} in
  the light of division II,'' in \emph{Appropriating Heidegger}, ed.
  James E. Faulconer and Mark A. Wrathall (Cambridge: Cambridge
  University Press, 2000), 155-174.} he concludes that there are in fact
two modes of authenticity. He writes, ``Heidegger clearly holds that
there is a form of understanding, of situations, on the one hand, and of
Dasein itself, on the other, that is superior to everyday
understanding.''\footnote{Dreyfus, ``Reinterpreting,'' 156-157.} The
majority of ``Reinterpreting'' is concerned with spelling out the
distinction between these two superior modes of being.\footnote{It may
  not be obvious to other critics that there are \emph{two} authentic
  kinds of understanding in \emph{Being and Time}. Indeed, the page
  referenced above (\emph{Being and Time}, 43) seems to make clear that
  there are only two. Dreyfus's view may be feasible if we believe that
  authenticity is on a kind of sliding scale---his strict bifurcation is
  curious. This paper is chiefly concerned with the most ``superior''
  kind of understanding; its criticisms of Heidegger and Dreyfus hold
  regardless of whether the bifurcation is present in \emph{Being and
  Time}. Also note that Dreyfus regards both authentic modes of being as
  ``superior'' even though Heidegger writes that inauthenticity is not
  inferior (\emph{Being and Time}, 43). In the second division of
  \emph{Being and Time}, though, Heidegger speaks about authenticity as
  an imperative for Dasein, affirming the sense that it is superior.} As
Dreyfus points out, the different modes of authentic being are the
result of one's understanding of a situation versus one's understanding
of one's being. What each of these looks like will be spelled out in the
following paragraphs.

For the sake of this paper, which chiefly investigates Dasein that
understands its being, we should also be aware of Heidegger's notions of
heritage and retrieval. Heidegger writes, ``The resoluteness in which
Dasein comes back to itself discloses the actual factical possibilities
of authentic existing \emph{in terms of the heritage} which that
resoluteness \emph{takes over} as thrown.''\footnote{Heidegger,
  \emph{Being and Time}, 383.} Dreyfus interprets this passage as saying
that the actor who merely understands their situation is not fully
authentic because they have not yet understood their possibilities of
action as those practices that have been given down to them by their
heritage, or performed by their ancestors. Heidegger continues,
``Resoluteness that comes back to itself and hands itself down then
becomes the retrieve of a possibility of existence that has been handed
down. \emph{Retrieve is explicit handing down}, that is, going back to
the possibilities of Da-sein that has been there.''\footnote{\emph{Id}.,
  385.} Retrieval is the consequence of Dasein's responding to its call
of conscience and repeating practices from the heritage's past. In
performing these practices, this actor's being is resolved into a
definite and fully authentic character. With this foundation for
Heidegger's philosophy, I will continue distinguishing between the two
kinds of authentic modes of being that Dreyfus finds in \emph{Being and
Time}.

The person who grasps their situation, but is not aware of their entire
being, is what Dreyfus refers to as the ``social virtuoso.'' They have
recognized the general rules of some social practice as contingent and
thrown, and stand as actualized within (versus abstracting from) their
situation. However, Dreyfus offers little sense of what this social
virtuoso could look like. I offer the example of a basketball all-star.
Imagine a ball handler dribbling up the court, trying to get around
their defender to drive to the basket for an easy shot. The ball handler
is successful in getting around the initial defender, but a second
defender runs towards the ball handler, blocking their clear path. The
ball handler now recognizes it as prudential to pass the ball to his
open teammate, but doing so would require making an impossibly abrupt
swivel of their body that could not be made because of the running
momentum. In an unprecedented move of improvisation, the virtuoso
driving to the basket passes the ball behind their back to their
teammate who makes the basket and wins them the game. Following this new
form of passing, the basketball league follows suit and players begin
passing behind the back, though the game of basketball is set-up the
same way with no new rules being added. The virtuoso acts so as to
guarantee their success, working under the rules of the game, but
getting beyond the fundamentals they learned to play ball in a more
effective way. The social virtuoso does not adhere to custom or
etiquette for its own sake and has eliminated general rules for success
in a practice, effectively re-contextualizing the practice they stand
in. Before continuing I want to make clear that the social virtuoso seen
above should be differentiated from an actor displaying mere competence,
who could be understood as an inauthentic actor. A competent actor will
successfully work within a practice without understanding it as a
definite situation, but instead recognizing it as a general set-up. Any
competent basketball player knows how to dribble, pass, and shoot; but
unlike the virtuoso, they will not respond with new approaches to these
basic elements.

Compared to the social virtuoso, the superior authentic Dasein who
understands their whole being acts as what Dreyfus calls a
``world-changer.'' Not only does this Dasein see the rules of some
social practice as arbitrary, but understands that the entirety of
social practices are a function of their ancestor's practices and those
earlier ancestors before them. But Heidegger also speaks of ``Dasein
hand{[}ing{]} itself down to itself,''\footnote{\emph{Id}., 384.} which
captures the sense in which Dasein finds its best possibility for acting
authentically in virtue of the structure of the heritage. In this way,
Dreyfus recognizes fully authentic Dasein as a being who sees that they
have been thrown into a heritage which has at the same time been handed
down. Dreyfus writes, ``Dasein can then act in such a way as to take
over or repeat the marginal practice in a new way and thus show a form
of life in which that marginal practice has become central and the
central practices have become marginal.''\footnote{Dreyfus,
  ``Reinterpreting,'' 167.} \footnote{Dreyfus is concerned with
  Heidegger's specific use of the German \emph{Wiederholen}, which is
  his technical term for the notion of retrieval that was seen above.
  Heidegger's notion of retrieval does concern traditional possibilities
  of action, particularly those possibilities which \emph{have been}
  (i.e. are not happening now). In this sense, the retrieved practices
  may be called ``marginal.''} What Dreyfus recognizes as fully
authentic Dasein, then, is one who in understanding their whole being
comes to see the exclusive possibility of authenticity in marginal
practices of its heritage's past. Where the social virtuoso has broken
from the public understanding in their field of expertise and
consequently updated it, the world-changer shifts to a fundamentally
changed way of life.

What Dreyfus's world-changer could amount to is open and unclear in
``Reinterpreting.''\footnote{The last section of Dreyfus's paper
  attempts to apply the social virtuoso and world-changer concepts to
  the juridical realm. But Dreyfus gives no concrete example of what
  marginal practice is being retrieved and no clue as to what heritage
  it is being pulled from besides the vague heritage of jurisprudence.
  He simply informs us how this character could exist.} I now turn to
examine Dreyfus's ``Comments on Jonathan Lear's `\emph{Radical Hope}'''
(``Comments'')\footnote{Hubert L. Dreyfus, ``Comments on Jonathan Lear's
  `Radical Hope','' \emph{Philosophical Studies} 144, no. 1 (May 2009):
  63--70, http://www.jstor.org/stable/27734426.} with the hope that this
character can be fleshed out and that we may find some insight into what
counts as the marginal practices they retrieve. Lear's \emph{Radical
Hope} is a book about the disintegration of the indigenous American Crow
culture following the tribe's forced migration by the American
government and the lessons it teaches about cultural destruction and
revival in general. Dreyfus looks to Heidegger to explain how cultural
revival could take place after a cultural destruction, or when a way of
life no longer has any meaning.\footnote{In Jonathan Lear's ``Response
  to Hubert Dreyfus and Nancy Sherman,'' he points out that Dreyfus
  maintains an incorrect reading of Lear's notion of cultural
  destruction. What Lear sees as a misreading is not pertinent to this
  paper because I am concerned with Dreyfus's comments on cultural
  revival (via Heideggerian world-changing) and not on the problem of
  cultural destruction.

  Jonathan Lear, ``Response to Hubert Dreyfus and Nancy Sherman,''
  \emph{Philosophical Studies} 144, no. 1 (May 2009): 81--93,
  https://link.springer.com/article/10.1007\%2Fs11098-009-9369-7.}
Dreyfus writes, ``In his later writings, Heidegger has a helpful answer
to how a cultural world could be radically reborn. He holds that, in
response to total world collapse one must become sensitive to marginal
practices.''\footnote{Dreyfus, ``Comments,'' 69.} \footnote{In this
  quote about a Heideggerian answer to collapse, Dreyfus finds material
  in the later writings of Heidegger (without citing any concrete
  source). Something is curious, though, because as we saw in
  ``Reinterpreting,'' Dreyfus finds reference to marginal practices in
  \emph{Being and Time}, which is generally not considered to be a later
  writing of Heidegger. The persisting element of vagueness in Dreyfus's
  account becomes reinforced without a clear citation.} It is in
response to cultural collapse that Dreyfus again introduces the
world-changer notion into the conversation.\footnote{Note that Dreyfus
  never explicitly uses the phrase ``world-changer'' in the Lear
  commentary, but his examples and word choice match the language and
  content from ``Reinterpreting.'' Moreover his Lear commentary was
  published some nine years after ``Reinterpreting,'' so Dreyfus's
  world-changer idea had already been formulated.}

In the Lear commentary, Dreyfus elects the Crow's return to farming
following their forced displacement as a case of a world-changing
innovation. The example reveals an additional qualification to the
world-changing character beyond an understanding of their being and a
resulting sensitivity to marginal practices. That is, the example shows
that the ``world'' being changed need not be a grandiose one; rather,
the ``world'' in world-changing refers to the Heideggerian sense of
heritage. Dreyfus follows Heidegger's framework for his discussion of
the Crow cultural revival, which is understood as a looking to the Crow
past for the Crow future. Viewing the Crow example through Heidegger's
framework, we see that the repetition of old marginal practices requires
that the reintroduced practices be ones that index a particular people.
That is to say, the Crow would not revive their own culture if they were
to adopt practices from another culture that merely resemble those of
their past. Revival is only authentic insofar as it is a revival of the
Crow's own farming practices. The way to confirm when and how a practice
properly indexes some people is unclear. It is not touched on by
Dreyfus. For Heidegger it is involved in the phenomenon of retrieval
following a response to one's call of conscience, but this too is not
clear. The world-changer now stands as someone who, in responding to
their call of conscience, changes their heritage by pulling on marginal
practices from the heritage's past.

Dreyfus's second example of a world-changer in the Lear commentary is
the real-world example of the Woodstock Festival. He cites Woodstock as
a case of a near world-changing that could offer us a new understanding
of being, but ultimately chalks it up as a failure. Woodstock rejected
mainstream concern and practice, opting instead for ``Pagan practices,
such as receptivity, enjoyment of nature, dancing, Dionysian ecstasy,
and non-exclusive love of one's neighbor.'' Dreyfus continues, informing
us of the social change that did not happen. He writes, ``The Woodstock
generation were not organized and total enough to sustain a
culture.''\footnote{Dreyfus, ``Comments,'' 69.} This latter quote adds
yet another qualification for the world-changer mode of being. The
world-changer does not change the world unless that culture is
sustained. That is, there must be some mechanism in place which is used
to reproduce the practices of that culture. We may wonder whether this
mechanism needs to be intrinsic to the culture or whether it can be
cultivated by infrastructure surrounding that culture (e.g. other
cultures?). This question, in turn, brings up more concerns about what
the world-changer looks like.

We have seen that the fully authentic actor pulls possibilities of
action from their heritage's past. But what is the scope of a heritage
or culture? In the case of the Crow people, it was relatively clear; but
in the Woodstock example, that clarity fades. Assuming they respond to
their calls of conscience, are the Woodstockers genuinely working within
\emph{their} heritage when they attempt to revive ``pagan practices''?
Are they actually changing a world (the heritage) when its practices
were obscured for hundreds of years, or are they introducing a new one?
Moreover, what does Dionysian ecstasy look like as a practice? How is
dancing in general a practice exclusive to one heritage, let alone the
pagan one? These questions and many others illustrate the thoroughgoing
vagueness of Dreyfus's notion of world-changer. Without explanation of
what a heritage is, when or how it is indexed by a practice, and what
can count as a marginal practice, the world-changer idea becomes more
trouble than it is worth, and vagueness threatens its rejection.

But, barring this problem of vagueness and continuing with the limited
criteria we do have, the Heideggerian notion of world-changer is
implicitly exclusionary, susceptible to internal tension, and
unjustifiably value-laden. I will demonstrate these issues through three
cases: one of artistic ingenuity, the second of reconciliation with
one's being through sex reassignment surgery, and the third of an actor
estranged from their heritage.

Consider a musical composer working in Vienna in the early twentieth
century. This revolutionary composer would go on to become definitively
world-changing following his conscious decision to compose according to
his invented twelve-tone technique.\footnote{The subject of the thought
  experiment is modeled after the real-world Viennese composer Arnold
  Schoenberg who invented the twelve-tone technique, appreciated a
  devoted following, and changed the musical world.} Let us imagine that
this new technique in composing is, by stipulation of the artist, a
conscientious rejection of the music heritage that has been handed down
to him. By conscientious rejection, we should understand that this
composer is fully aware of his being, responding to his own call of
conscience, as well as the esoteric practices waiting to be revived
within his Western music heritage. In breaking with the heritage before
him he has freed himself up to explore new possibilities in music and
expression in general. Allow me to also claim that this composer enjoyed
a devoted following and respect from composers around him. That is to
say, there was a mechanism in place for this composer's invention to
reach significant audiences amongst both musical lay people and musical
highbrows, thereby guaranteeing that the twelve-tone practice of
composition is sustained. This composer, despite responding to an
understanding of his own being and creating a sustained change within
that heritage, cannot be considered fully authentic on either
Heidegger's picture or Dreyfus's adaptation. Precisely because this
composer has not retrieved the marginal practices and the larger
heritage he is aware of, but instead rejects them, he is excluded from
an authentic character. Beyond the exclusionary nature of the
Heideggerian notion of authenticity and its appropriation by Dreyfus we
also see that the idea is susceptible to an internal tension.
Authenticity requires that Dasein responds to its individual call of
conscience which would result in a retrieval from its heritage. But we
have an example in the composer of someone whose response to his call of
conscience---his only means of becoming authentic---requires a rejection
of his heritage. Through the example of the composer, we see that the
exclusion of new ways of being and the tension between its call of
conscience and its retrieval follow as logical consequences on the
Heideggerian view of authenticity.

A response on behalf of Dreyfus or Heidegger would hold that I have
mischaracterized authentic retrieval of the marginal practices. This
objector points to \emph{Being and Time}, where Heidegger writes that
retrieval means a return ``to traditional possibilities, although not
necessarily \emph{as} traditional ones.''\footnote{Heidegger,
  \emph{Being and Time}, 383.} The objector contends that in fulfilling
a possibility overlooked by the heritage, the composer is not only
sensitive to marginal practices, but the composer's invention only finds
a significance because of the musical background it rests on. But the
objector misunderstands that what is meant by ``\emph{as}'' in the quote
above could not encompass the negative possibility, i.e. the possibility
never before acted on that the composer invents. Instead, for Heidegger,
the ``as,'' serves to emphasize that the authentic retrieving by an
actor is not performed merely for the sake of repeating the past.
Rather, an actor who responds to their call of conscience retrieves
marginal practices of their heritage because their character is thereby
realized. This means both freeing up the actor from the endless
possibilities that could drive them to nothingness as well as rescuing
them from the ``closest possibilities offering themselves---those of
comfort, shirking and taking things easy.''\footnote{\emph{Id}., 384.}
Heidegger leaves us wanting for examples of an authentic retrieval, but
the objector finds no support in Dreyfus's examples. The Crow revive
their own farming practices and Dreyfus claims that Woodstock retrieved
pagan practices---in each case, these are actual possibilities performed
in the past. The composer, on the other hand, does indeed act on a
possibility that was always available to the tradition, though never
before acted on. And the objector is right to point out that their
invention finds significance (at least in part) because of the heritage
it rejects or moves beyond. However, the invention's finding
significance in light of the heritage is not equal to saying that the
invention is a practice retrieved from the past. The invention is just
that: an invention. As such it could not be a retrieval in the
Heideggerian sense. Here the troubling nature of the account becomes
evident again: rejecting or moving beyond the heritage excludes that
actor from being authentic. The exclusionary nature of Dreyfus and
Heidegger becomes more apparent and severe in cases where the actor
lacks an obvious heritage to retrieve marginal practices from.

Next consider the case of a transgender woman who undergoes a gender
confirmation surgery in 1950s America.\footnote{This second illustration
  is modeled after the real-world example of Christine Jorgensen who was
  the first American to undergo gender confirmation surgery. She
  continued to advocate for transgender rights, helping to catalyze a
  movement that would change our social world.} She is one of the first
persons in the world to have the operation, partly because her
generation is the first to enjoy the surgical possibility. We can
imagine that many before her felt limited by the gender assigned to them
at birth. But she found herself in a unique position, not only in
understanding this mischaracterization about her being, in addition to
understanding her being in total, but that she could respond to her call
of conscience to accommodate this understanding of her being by
undergoing the operation. After her operation, she becomes a public
icon, spreading awareness of the new possibility, and working to create
a respect for the growing transgender community in the greater culture.
In this way, gender confirmation becomes an operation that is sustained
and adopted in larger numbers. Despite the presence of transgender and
non-binary peoples throughout history, it is far from obvious that there
was a definite heritage for this woman to refer to---here, the problem
of vagueness rears its ugly head yet again. Also note that the practice
of gender confirmation surgery (as contrasted with castration) was not a
possibility before the mid-twentieth century, so there was no practice
to retrieve. Therefore, not only was gender confirmation surgery not a
marginal practice, but there was no heritage for the surgery to be a
marginal practice of. On Heidegger's account and Dreyfus's utilization
thereof, this world-changing figure could not be characterized as
authentic. This person acts in a way that rejects a heritage handing
down the constricting binary notion of gender. For her, to not undergo
the surgery is not only to be oppressed in her being, but is to be
prevented from responding to her call of conscience and her actual
possibility of authenticity, despite being excluded on the Heideggerian
account.

Consider one last example of how an actor's authentic retrieval from a
heritage could be an impossibility. In the exposition above, we saw that
Dreyfus offered a response to the problem of cultural collapse. But, how
could one revive a culture whose practices have become meaningless?
Dreyfus says that the authentic actor can retrieve marginal practices
``precisely because such practices would not have been central to the
meaning of the past way of life {[}so{]} they could survive the
collapse\ldots{} Radical world rebirth can take place if and only if one
can make some marginal practices central.''\footnote{Dreyfus,
  ``Comments,'' 69.} Under the neat example of the Crow people, who
retrieve their settled farming practices, Dreyfus's suggestion appears
substantiated. However, he conveniently leaves out the all too plausible
case that some actor becomes completely estranged from their heritage
which has been utterly destroyed. There are many ways one could flesh
out this thought experiment, for example, a lab technician at a
cryogenic lab becomes frozen for a thousand years only to wake up and
see that the world is ruled by two corporate cultures; a child is robbed
from a homeland buried under colonial infrastructure, removing any
possibility of their return to traditional practices; etc. Even if there
are practices in the adoptive cultures that resemble those of the
destroyed heritages, engaging in these practices could not count as the
actor's being authentic. One reason is that they have not
\emph{retrieved} these practices out of the margins, but just as
important is that the marginal practices to be retrieved are those that
index the descendants of a heritage. In this example, though, the
indices die with the practices and the heritage. So, a consequence of
the rigid view posited by Heidegger and appropriated by Dreyfus is that
these actors, estranged from their totally unintelligible heritage, have
no chance of reaching authenticity.

Beyond the thick vagueness of the world-changer idea that was pointed
out above, I believe there are two general takeaways that these examples
demonstrate. (Also note that these critiques hold regardless of the
concept's vagueness.) Firstly, the idea of locating authenticity in
heritage is exclusionary and undermining to the theory because of its
internal tension. With the increased prominence of critical race theory,
gender studies, and movements that shine a light on underrepresented
peoples, we have been confronted with new and challenging questions
about identity and being. The emphasis on heritage we have seen would
have us disregard a concept like intersectionality, which challenges the
dated thought that there is a clear-cut heritage for any individual to
retrieve practices from. The Heideggerian picture can lead us to dismiss
tough economic and political questions regarding cultural
cross-pollination and appropriation, if we are to believe that
authenticity requires a return to our own heritage. We may wonder when
it is appropriate and beneficial for ourselves and others to champion
and adopt the practices and ideals of heritages outside of our own.
Dreyfus's Woodstock example implies an advocacy for this kind of
retrieval, one that is both a return to a distant past and at the same
time a rejection of the immediate conventional practices and values. In
addition to its problematic exclusionary nature, the Heideggerian view
suffers susceptibility to an internal tension. The view says that the
authentic must respond to their conscience, but must also retrieve from
their heritage---but of course retrieving from the heritage does not
provide a real possibility to find authenticity for those whose heritage
is opposed to their being. More work must be done to parse out its
significance in the greater whole, but it seems that this picture of
authenticity would be strengthened by abandoning the requisite return to
heritage. In that case, the persons presented above have a well-deserved
chance at authenticity.

The second takeaway from the examples is that the Heideggerian view
implicitly contains unjustified hierarchical valuations. This view
privileges the practices of the past, touting a return as a means to
avoid idleness, without acknowledging the potential limitations of one's
heritage. It holds that any desire to work towards a future that breaks
with practices of a heritage's past is inauthentic. But we should not
fault members of the Crow Nation who assimilated into a different
culture just on account of their hope for finding a new, different being
from their ancestors. This unjustified hierarchy and normativity is
embedded in Dreyfus's use of the phrases ``world-changer'' and ``social
virtuoso.'' Transgender people responding to their own calls of
conscience to undergo gender confirmation surgery today should not be
considered any less authentic than others because they have not
individually ``changed the world.'' Beyond the implicit hierarchical
nature, the expression ``world-changing'' does not accurately capture
what Dreyfus wants. We saw that there are many cases of people who make
conscious turns from what was handed to them to become undeniably
world-changing. The solution is not to add additional strata or
categories of authenticity---e.g. a world-changer that does not pull
from heritage---but to recognize that the view dressed by this language
is a limiting one. It should be updated to accommodate a more diverse
picture of action and being.

The failures of the aspect of traditional retrieval follow as
consequences of the greater methodology of \emph{Being and Time}.
Despite his thoroughgoing phenomenological genius, Heidegger's desire
for a personal, solitary investigation into the question of being
removes his own being from the world he finds himself jarringly thrown
into. In attempting to critique and transcend the humanistic
philosophical tradition descended from Plato, Heidegger only reaffirms
an egoistic and culturally chauvinistic character. The lack of
meaningful dialectic and diversity in this philosophy's method can set
one up to overlook a heritage's oppressive aspects and practices.
Unfortunately, this rung true for Heidegger the man. In lieu of this
rigid view, we should be more open to accept a diverse and radically
shifting theory of action that does not locate authenticity solely in
the retrieval of practices of the past. Rather, we may try to understand
authenticity as acting on something missing from the world or in one's
own being. The view coming from Heidegger and Dreyfus maintained an
inkling of this picture, suggesting that the need for an authentic
character drove people to revive those practices that were missing. But
just as Heidegger does not want actors to retrieve historical practices
just for the sake of their being historical, we should not act on what
is missing just for the sake of being different, but also as a sort of
necessity. So, this picture sees the innovative composer and the
transgender woman as acting authentically, both as a necessity of their
being, but also to create a necessary difference that moves the world.
It also provides hope to the estranged person who can find authenticity
despite the tragedy which befell their people and home. The source of
understanding that is missing from one's being and the world could come
not only from personal meditations on one's own being, but through the
added dimension of discourse.

\section*{References}

\begin{itemize}[label={},itemindent=-2em,leftmargin=2em]
	\item Dreyfus, Hubert L. ``Could anything be more intelligible than everyday intelligibility? Reinterpreting division I of \emph{Being and Time} in
  the light of division II,'' in \emph{Appropriating Heidegger}. Edited by James E. Faulconer and Mark A. Wrathall. Cambridge: Cambridge University Press, 2000, pp. 155-174.
	\item Dreyfus, Hubert L. ``Comments on Jonathan Lear's `Radical Hope','' \emph{Philosophical Studies} 144, no. 1 (May 2009): 63--70.
	\item Heidegger, Martin. \emph{Being and Time}. Translated by Joan Stambaugh. Albany: SUNY Press, 2010. 
\end{itemize}