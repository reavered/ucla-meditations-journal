\chapter*{Moral Residues}
Imagine that I promised John to deliver widgets to his factory but later decided not to do so simply because I felt lazy. As a consequence of my behavior, John’s business suffered a tremendous loss. Surely morality would call on me to do certain things to make up for the broken promise. In Chapters 3 and 12 of The Realm of Rights1 Thomson notices the existence of such moral norms in certain2 cases where an agent fails to accord with her words. She calls such moral norms ``moral residues''. Thomson contends that it is necessary to provide an account for the existence of moral residues in order to fill in the gap between the fact of breaking a promise and the fact of the existence of moral residues. By examining intuitive cases of moral residues, Thomson presents her account for the existence of moral residues in those cases, which states that moral residues result from failing to fulfill a certain type of claims.

Depending on the specific word-giving cases being examined, there might be different accounts of moral residues. In this paper, I will limit my discussion of moral residues to certain3 cases of broken promises. I will first explain Thomson’s account for the existence of moral residues by applying it to certain4 cases of an agent’s failing to fulfill a promise. I will then object to this explanation 1) by pointing out that in some of such cases moral residues are in place even when there is no Thomsonian claim, and 2) by arguing that the Thomsonian account does not bridge the gap between the non-normative and the normative. I will instead provide what I regard as the best account of moral residues in certain cases of an agent failing to fulfill a promise by appealing to what it means to make and break a promise. Finally, I will contrast my account of moral residues with Thomson’s to show that my account avoids the problems that Thomson’s account faces.