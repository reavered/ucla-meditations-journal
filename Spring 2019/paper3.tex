\paragraph{\small Abstract.}

{
\footnotesize
What is `art'? Which things count as `political'? Different parties
disagree endlessly on the answers to these questions. W. B. Gallie holds
that even though any two of these parties may each be convinced that the
other is misusing the term in question, all of the parties' clashing
uses may be correct at once. In this paper, I propose that
`(un-)Americanism' is a term of this kind, marking it as what Gallie
calls an `essentially contested concept.' To argue this, I first look at
contemporary and historical uses of `American,' `un-American,' and
related terms. I then turn to Gallie's standards for determining whether
a concept is essentially contested, standards which I slightly modify.
This allows me to conclude that `(un-)Americanism' is indeed such a
concept, which helps explain why Democrats and Republicans each see
themselves as `American' and their counterparts as `un-American.' To
explain this disagreement further, I finally turn to Hans Sluga's
analysis of politics as a `hyper-complex system.' Sluga argues that the
political system is far too complex for observers, and even the
politicians within it, to thoroughly understand it---a great deal of
uncertainty will always remain about how and why the system produces the
results that it does. I find that this uncertainty also plays a key role
in our disagreement over what is and is not `American.'\footnote{Thanks to Hans Sluga for helping me focus the task of
  this paper and for introducing me to the two key texts it draws on
  (one of which is his work) in the seminar for which I originally wrote
  it. Thanks also to the \emph{Meditations} editorial board for
  providing a great deal of help with revisions.}
}

\section{The Backdrop: Competing Notions of `Americanism' in the
21\textsuperscript{st} Century}

During the 2008 general election, three Republican politicians made
headlines by publicly offering their thoughts on who was `pro-America'
or a `real American'---and, crucially, who was not. On \emph{Hardball},
Representative Michele Bachmann said of then-Senator Obama, ``I am very
concerned that he may have anti-American views'' and implored the media
to thoroughly investigate ``the views of the people in Congress and find
out are they pro-America, or anti-America.''\footnote{The Politico.
  ``Bachmann Calls For `Penetrating Expose' On `Anti-Americans' In
  Congress.'' (\emph{CBS News}, 17 November 2008.)}
At a rally, Sarah Palin described small towns as part of ``the real
America'': the ``hardworking, very patriotic {[}\ldots{}{]} pro-America
areas of this great nation {[}\ldots{}{]} where we find the kindness and
the goodness and the courage of everyday Americans. Those who are
running our factories and teaching our kids and growing our food and are
fighting our wars for us.''\footnote{Stein, Sam. ``Palin Explains What
  Parts Of Country Not `Pro-America.''' (\emph{Huffington Post}, 17
  November 2008.)}
At another rally, Representative Robin Hayes called Palin ``a great
American,'' declaring by contrast that ``liberals hate real Americans
that work and accomplish and achieve and believe in God.''\footnote{Layton,
  Lindsey. ``Palin Apologizes for 'Real America' Comments.'' (\textit{Washington Post}, 22 October 2008.
)}

Eight years later, many prominent Democrats advanced a different view of
what was and was not `American.' During the spring primaries, Bernie
Sanders described both Sheriff Joe Arpaio's immigration enforcement
tactics and Wisconsin's voter ID laws as ``un-American,'' President
Obama called Donald Trump and Ted Cruz's anti-Muslim rhetoric ``wrong
and un-American,'' and talk-show host Jimmy Kimmel asked Trump of his
proposed Muslim ban, ``Isn't it un-American and wrong to discriminate
against people based on their religion?''\footnote{Wetzler, Libby.
  ``Un-American and Unclear: What Exactly Does it Mean to be
  Un-American?'' (\emph{Fordham Political Review}, 27 April 2016.)}
That July, at a Democratic National Convention centered on casting
progressive values as the American antidote to un-American Trumpism,
Secretary Clinton delivered this rejoinder to Trump: ``Americans don't
say, `I alone can fix it.' We say, `We'll fix it together.'''\footnote{Zorn,
  Eric. ``Column: No need to distort Trump's and Clinton's words. The
  truth is bad enough.'' (\textit{Chicago Tribune}, 2 August 2016.
 )}
In her convention speech, Michelle Obama posited diversity and
collective striving as the core American values that have always
animated the country's citizens, saying, ``I want a president
{[}\ldots{}{]} who truly believes in the vision that our Founders put
forth {[}\ldots{}{]} that we are all created equal, each a beloved part
of the great American story. And when crisis hits, we don't turn against
each other. No, we listen to each other, we lean on each
other.''\footnote{Washington {Post} Staff. ``Transcript: Read
  Michelle Obama's full speech from the 2016 DNC.'' (\textit{Washington Post}, 26 July 2016.)}
Her husband's speech also framed these as defining American values
threatened by Trump, saying ``the fanning of resentment, and blame, and
anger, and hate {[}\ldots{}{]} is not the America I know. The America I
know is full of courage, and optimism, and ingenuity. The America I know
is decent and generous.''\footnote{\emph{ABC News}. ``FULL TEXT:
  President Barack Obama's 2016 Democratic National Convention Speech.''
  (28 July 2016.)}
``What makes us American, what makes us patriots, is what's in {[}our
hearts{]},'' he added later, crediting the country's integration of
diverse cultures and ability to draw entrepreneurs from all over the
world to the American heart. ``That's why anyone who threatens our
values, whether fascists or communists or jihadists or homegrown
demagogues, will always fail in the end. That is America.'' The
Republican Party, then, does not have a monopoly on this practice; there
are plenty of examples of both parties' members attempting to claim the
`American' high ground for themselves.

\section{A Question and a Roadmap for Addressing
It}\label{a-question-and-a-roadmap-for-addressing-it}

How are we to explain these instances wherein each major political party
in the United States brands itself as American and its opposition as
un-American?\footnote{When I refer to the actions or intentions of a
  political party as a whole in this paper, I do not mean to literally
  suggest that either party has a definable unity of action or purpose.
  Rather, I intend to refer to a convenient generalization I am
  abstracting from some combination of (a) views commonly expressed by
  party actors and (b) features I take to be salient and relatively
  consistent across the party's internal operations. (See my discussion
  of `hyper-complex systems' in section 5.)}\textsuperscript{,}
\footnote{Of course, political parties and their representatives have
  incentives to smear each other---often with each side employing the
  same smears against the other, as constantly seen, for instance, in
  campaigns where each candidate accuses the other of supporting
  burdensome tax raises---while propping themselves up, regardless of
  facts and sound arguments. While this surely plays a role in the
  explanation, I do not believe it is exhaustive; I think there is some
  underlying logic to be parsed and that in any case it is worth
  investigating the reasons that politicians from both parties find
  their supporters persuadable to opposite conclusions on who or what is
  truly American. Further, I do not mean this to imply any sort of
  general equivalence between the two parties---while I take it to be
  readily apparent that conservative political figures and commentators
  are more prone to otherizing and making bad-faith attacks than their
  liberal counterparts, this disparity is not what I aim to analyze
  here.} Suppose we model each party's argument as follows (though this
particular construction is not essential to my analysis):

\begin{enumerate}
\def\labelenumi{(\arabic{enumi})}
\item
  \begin{quote}
  A resident of the United States is a real American (or pro-America) if
  and only if she subscribes to the set of values X, possesses the set
  of traits Y, and performs the set of actions Z.
  \end{quote}
\item
  \begin{quote}
  Members of our party (generally) subscribe to X, possess Y, and
  perform Z, while members of the other major party (generally) do not.
  Thus,
  \end{quote}
\item
  \begin{quote}
  Members of our party are (generally) real Americans (or pro-America),
  while members of the other party (generally) are not (1, 2).
  \end{quote}
\end{enumerate}

On this model, does the mutual contradiction of each side's holding the
conclusion in (3) true from its perspective owe more to a disagreement
on the contents of sets X, Y, or Z in the first premise or a
disagreement over which values, traits, and actions characterize the
respective parties' members in the second premise?\footnote{More
  specifically, on which of these two issues is the typical level of
  disagreement greater \emph{between} the two parties than \emph{within}
  them?} That is, does the dispute center more on a contested definition
of `(un-/pro-/anti-)American' or on the stark differences between how
each party characterizes itself and how its opponents characterize it?

To evaluate this question, I will first examine the dispute over the
definition of `American,' beginning in section 3 with a brief history of
the `American versus un-American' dichotomy, followed by a rough
formulation of how the major parties use these terms today and an
examination in section 4 of whether `(un-)American' qualifies as what
Gallie calls an `essentially contested concept.' Finally, I will move in
section 5 from conflict over the definition itself to what its
conflicting applications say about the parties' views of themselves and
each other and how these clashing views are defined by an inescapable
uncertainty from which my analysis is not immune.

\section{A History of Contradicting
Uses}\label{a-history-of-contradicting-uses}

My first step in analyzing the contested use of these terms will be to
seek some historical context. In the middle of Obama's aforementioned
convention speech, he said the notion that ``there's a `real America'
out there that must be restored'' had ``been peddled by politicians for
a long time -- probably from the start of our Republic.''\footnote{Ibid.}\textsuperscript{,}
\footnote{If this tactic has always been favored by American politicians
  as Obama suggests, that would seem to mark it as arguably an
  `American' practice. I expand a bit on this notion of `Americanism' as
  founded descriptively on historical practices associated with the
  United States in sections 4 and 5 (particularly in footnotes 38 and
  41).} While this statement seems painfully un-self-aware in a speech
casting Clinton's vision as one rooted in the true America and its real
American values, with Trumpism as the antithesis of that vision, it
appears to be historically accurate. According to Beverly Gage, a Yale
professor of American political history, charges of un-Americanism have
indeed been thrown around for nearly as long as constitutional democracy
has existed in the United States.\footnote{Gage, Beverly. ``How
  `Un-American' Became the Political Insult of the Moment.'' (\emph{The
  New York}

  \emph{Times Magazine}, 21 March 2017.)}
The term `un-American' could be found in print as early as 1818, when a
book passage employed it to denounce how much was spent on Capitol
Building repairs after the War of 1812.\footnote{Wetzler.} Less than a
century later, segments of the American public were already applying the
term to a wide range of targets with no clear unifying principle for its
use, leading the St. Louis \emph{Post-Dispatch} to lament in 1909 that
``Americans are very fond of classing as un-American anything they don't
like.''\footnote{Gage.} Illustrating this point, the piece complained
that not only social alcohol consumption and boycotts, but the opposing
practices of prohibition and employers' intimidation of workers, were
among a long list of things decried by their respective opponents as
`un-American.'

From there, prominent figures capitalized on the notion of
un-Americanism to serve greater political ends. Seeking to quash dissent
and whip up patriotic fervor behind the war effort, President Wilson
demanded ``100 percent Americanism'' during World War I, a call which
served to demonize both dissenters and German immigrants---helping to
justify the former's imprisonment under the Espionage Act\footnote{Ibid.}
and the latter's mandatory registration, interrogation, and
internment\footnote{Krammer, Arnold. \emph{Undue Process: The Untold
  Story of America's German Alien Internees.} (Lanham, MD: Rowman \&
  Littlefield, 1997.) 14.} under a pair of 1917 executive
orders.\footnote{\textit{The New York Times}. ``Gregory Defines Alien
  Regulations.'' (2 February 1918.)}
In the 1920s, calls for `Americanism' bolstered anti-immigration
policies and expansion of the Ku Klux Klan.\footnote{Gage.} Yet even as
this rhetoric was used to justify both wartime and post-war restrictions
on speech and immigration, `un-American' enjoyed a period of
near-antonymous popular use as a derisive label for such restrictions
among members of a growing pro-civil liberties opposition before the
1930s saw politicians increasingly monopolize the term toward repressive
ends once again. This culminated in the 1938 establishment of the House
Committee on Un-American Activities (HUAC) in likely the most famous use
of the term to date, as HUAC's focus turned increasingly toward
suppressing the American Left under the guise of neutralizing an
anti-American communist threat---an effort which the committee joined
much of the Senate and executive branch in ramping up during the
McCarthy era of the late 1940s and 1950s.\footnote{Ibid.} It thus seems
no coincidence that `un-American' appeared at its greatest-ever
frequency\footnote{As of mid-2016, at least.} in English-language books
in 1949.\footnote{Wetzler.} It is hard to say, though, what portion of
its appearances in that year's literature were pro- rather than
anti-McCarthyism; by the previous year, the government's controlling
grip on the term had loosened again to the point that an FBI memo had
mentioned a growing trend of citizens coming to view ``the House
Committee on Un-American Activities as being un-American,
itself.''\footnote{Gage, quoting directly from the FBI memo.}

It is not new, then, for opposing parties to invoke the
`American'/`un-American' binary to their own contradictory ends. Rather,
it is such conflicting usage that seems to define the concept's entire
history, with no discernible origin in anything like a concrete,
agreed-upon set of criteria designating someone or something as American
or un-American. Therefore, I find that we likely lack a stable
definition of this concept not only over time, or within our own time,
but within every era in which it has been in popular use. There does
seem to be a through-line, though, of historical and contemporary users
alike invoking the concept as an in-group rallying cry around a
selective construal of national identity that serves each user's favored
cause. I see this as the most coherent way to unify everything from
Wilson's anti-free speech rhetoric to Sanders' pro-voting rights
comments under the same umbrella---in each instance, I take the user's
aim to be that of rallying followers behind a supposed, desirable,
`American' group and in opposition to a supposed, undesirable,
`un-American' group (and this aim might even commonly extend to stoking
in-group fear, and fear of persecution, by implying that everyone who
fails to fully support the former group is necessarily in the latter,
though this implication is much more easily located in uses such as
Wilson's advocacy of ``100 percent Americanism'' than examples like
Sanders' soundbites about ``un-American'' practices). The `American'
group, it seems, is suggested to properly uphold something between core
traditions that have defined American society since the country's
founding and the core values to which America has always aspired, or at
least to which it \emph{should} aspire;\footnote{The variants of the
  suggestion I put forth here move increasingly, in the order I list
  them, from a mostly descriptive to a largely prescriptive slant, an
  issue I expand on in footnotes 38 and 41.} the `un-American' group,
meanwhile, is painted as opposing these traditions or values. In this
way, the rhetorical appeal to Americanism, un-Americanism, or any of the
aforementioned related terms seems uniformly to consist in an informal
logical fallacy: namely, an \emph{ad hominem}. Specifically, uses of
`un-American' tend to exemplify the more common, narrower meaning of
\emph{ad hominem} as a logically irrelevant attack on an opponent's
character rather than the opponent's argument itself while uses of both
`un-American' and (the relevant, opposing sense of) `American' typically
fall under the less common, broader-ranging definition of the fallacy as
an appeal to an audience's prejudices or emotions rather than reason.

\section{Is `(un-)American' an Essentially Contested
Concept?}\label{is-un-american-an-essentially-contested-concept}

Let us now shift focus from the commonalities between historical and
contemporary usage of this concept to an assessment of whether these
usages have marked the concept as `essentially contested' on W. B.
Gallie's model. In a paper titled ``Essentially Contested Concepts,'' he
analyzes the phenomenon wherein different parties' conflicting, yet
simultaneously correct, uses of one and the same concept guarantees
interminable conflicts between these parties over its correct
usage.\footnote{Gallie, W. B. ``Essentially Contested Concepts.''
  (Oxford University Press. \emph{Proceedings of the }

  \emph{Aristotelian Society}, New Series, Vol. 56 (1955 1956), pp.
  167-198. )} For Gallie, prominent
examples of such essentially contested concepts include `democracy,' a
`work of art,' or the `Christian doctrine'---different individuals,
political groups, religious communities, and so forth maintain
irreconcilably different ideas of what falls under each of these
concepts' headings (not despite, but rather owing to, each party's
competent usage of these terms, as the ``standard general use'' of any
such concept is composed of the totality of these varied, clashing
competent usages).\footnote{Gallie, 168-169.} Hence, whether ballot
initiatives---which are arguably the most direct means of founding
policy on voters' will, but provide openings for powerful corporations
to manipulate underinformed voters into enacting pro-corporate policies
against these voters' own interests---are more or less democratic than
the indirect, `representative democracy' model of the legislative
process, or whether either or both count as democratic at all, is the
subject of endless dispute between groups maintaining different (but
nonetheless proper) definitions of `democracy.' Similarly, whether dime
novels, superhero movies, edible arrangements, or internet pornography
count as `art' is irreconcilably contested by plenty of opposing parties
who possess a high degree of artistic expertise.

Gallie's search for the root of what makes a concept essentially
contested ultimately leads him to develop four central criteria (and
three secondary ones) that serve both to illuminate what gives rise to a
concept's essentially contested nature and as a guide to sorting
concepts that are essentially contested from ones that are not. Under
his first criterion, an essentially contested concept is necessarily
`appraisive': for someone or something to be categorizable under such a
concept is for them to have achieved something of perceived
value.\footnote{Gallie, 171.} I do not find this criterion
necessary---for instance, I agree with Gallie that what does or does not
fall under the concept of art is essentially contested, but it seems
that under all but the strictest interpretations of `art,' it is not
necessarily a valued achievement to create an instance of it. After all,
most would agree the world has plenty of both terrible art and terrible
artists, but they still count as instances of those respective concepts
regardless of whether they have achieved anything of value. Even if this
criterion is not necessary, the concept picked out by the term
`American' (and such related terms as `pro-America' and `real American')
as used in this paper's examples straightforwardly meets it. In each of
my examples, contemporary and historical alike, the term is associated
with some cluster of values, traits, or actions held as ideal and more
specifically implied to demonstrate moral goodness. The people and
things classed as `un-American' (or `anti-American'), meanwhile, are
consistently so categorized in the above examples to posit them as
falling outside the `American' concept's boundaries and thereby failing
to achieve those valued American ideals.

Gallie's second criterion requires that the achievement picked out by
the first criterion possess an internally complex nature---not one that
can be assessed by simple, objective measurements.\footnote{Gallie,
  171-172.} In contrast to his first criterion, the requirement of
internal complexity does strike me as necessary;\footnote{That is, I
  find this, and the subsequent criteria I will admit as necessary, to
  at least be necessary to the extent one disregards any portions that
  echo the first criterion's specification that the achievement must be
  perceived as valuable (per my objection to the first criterion above).}
it is hard to fathom how any concept whose members are determined by
simple measurements (e.g. Presidents of the United States, winning teams
determined by official score, most-viewed television programs) could be
ceaselessly disputed by different factions unless no more than one such
faction competently uses the term. `Americanism' also satisfies this
criterion in contemporary usage, as everyone invoking the concept
appears to use an array of subjective metrics for achieving Americanism,
metrics that even individually tend to be complex and difficult to
measure. To qualify as a real American in Hayes' eyes, one must work,
achieve, and believe in God. Reasonable people would disagree on how to
measure each of those things. (Is grinding away as a stand-up comedian
sufficiently hard work? What about homemaking? Does one need to work
full-time? Are the achievements that count monetary, personal, or only
ones which benefit one's community---and in any case, how do we assess
what achievements are sufficiently large and well-earned? Finally, what
are the restrictions on belief in God? Does only the Christian God, or
perhaps the God worshipped by a specific subset of a specific Protestant
denomination, count? Is one disqualified if their faith sometimes
wavers, or they only worship because they find Pascal's Wager
persuasive?) Palin's criteria for being ``pro-America''---or part of
``the real America''---include ``kindness,'' ``goodness,'' and
``courage,'' and she offers up ``protecting the virtues of freedom'' as
one way to qualify.\footnote{Stein.} These, too, are all subjective
qualities which are difficult to measure. In yet another multifaceted,
subjective, and measurement-resistant conception, President and First
Lady Obama and Secretary Clinton jointly posit the love of neighbor,
embracement of diversity, commitment to equality, rejection of both fear
and authoritarianism, and spirit of shared struggle and sacrifice, as
well as decency, generosity, and the simultaneous acknowledgement of
America's past moral failings and belief in---and striving
for---continued progress in overcoming them, as essential features of
Americanism.

These examples demonstrate the concept's fulfillment of not just the
second but the third criterion, on which any account of what makes the
relevant achievement valuable\footnote{In keeping with my resistance to
  the first criterion, I would prefer to weaken ``what makes the
  relevant achievement valuable'' to ``what makes fulfilling the concept
  meaningful'' (but the former better paraphrases Gallie).} must
reference some varied set of components that mark an individual (or
group, object, or event) as achieving it.\footnote{Gallie, 172.} On this
criterion, it must also be true of any essentially contested concept
that when one first attempts to formulate the mixture of components
sufficient to satisfy it, many different potential orders of
prioritizing these components must appear plausible---i.e., it cannot be
clear right from the outset that there is precisely one correct formula
for instantiating the concept. This criterion, too, strikes me as
plausibly necessary: if there is one immediately obvious formula for
someone or something's achieving the status conferred by a concept, it
seems that disputes over what attains that status would likely be
resolvable by simply following this consensus formula. Take, for
instance, a complex board game whose concept of a `winner' is fixed by
comparing players' performance on some weighted hierarchy of a variety
of achievements that may be attained during play. Even if this case of a
valued achievement of an internally complex nature satisfies the first
two criteria, the obviousness of which precise system must be employed
to weigh the factors that determine the winner means that no two players
with (1) a proper concept of `winning' this game, and (2) matching
records of the objective facts as to which players earned which
achievements over the course of the game, will disagree on who has won
(other than, perhaps, in a temporary dispute which is easily resolved by
a joint, objective, step-by-step process of re-checking their execution
of the formula). In contrast, there is no singularly clear, correct way
of weighting generosity versus work ethic and religious faith, or
determining the respective priority granted to commitment to equality
versus love of neighbor and clear-eyed striving for the moral betterment
of American society. `(Un-)Americanism' therefore meets this criterion
as well.

The above comprise three of the four criteria Gallie holds to be the
most vital prerequisites to a concept's being essentially
contested.\footnote{Ibid.} The fourth and final of these is that the
character of an essentially contested concept's defining achievement
must be liable to change substantially in ways that cannot be chosen or
anticipated beforehand. As with the second and third criteria, I find
this necessary. If the standards are fixed with certainty indefinitely
into the future, it is unclear how they might be irreparably disputed
between rational observers and their contemporaries within any given
time. If potential changes in the standards dictating when a concept
obtains can be prescribed or predicted before they occur, it seems the
standards must be so easily observable and stable as to allow their
deliberate manipulation or a clear view of their trajectory, in which
case it again seems doubtful that they will be complex or unstable
enough to generate endless disputes among those with proper
understandings of the concept. However, I suspect Gallie would accept a
couple of small caveats here. First, what tends to drive the shift over
time in which array of competing characterizations of a concept
dominates the debate between conflicting groups of its competent users
is an accumulation of individual people
choosing---\emph{prescriptively}---new preferences that beget shifts in
dominant social attitudes as people increasingly adopt them. Second,
while most would-be prognosticators' attempts to predict future social
trends are liable to fail,\footnote{Owing largely to a set of epistemic
  limitations which includes some of those listed in section 5.} shrewd
analysts do sometimes extrapolate from their knowledge of historical and
present trends and circumstances to successfully predict probable social
changes.

Allowing these two caveats, `Americanism' meets this criterion. In a
country founded on slavery that once denied the vote to everyone but
white, male landowners, the bounds of debate over what is `American'
have shifted enough among competent users that Bernie Sanders now
plausibly seems to represent a popular perspective when he calls
Republican tactics to suppress the black vote ``un-American.'' Moreover,
throughout most of the country's history, dominant conceptions of
American identity have seesawed back and forth between (a) America as a
nation of immigrants and (b) periodic appeals to American identity as a
basis for excluding various groups of foreign origin. We can see the
former still on display in section 1's Obama and Kimmel quotes and the
latter echoing from the late 1910s and 1920s through Trump
administration officials' (transparently racist, and contrary to violent
crime statistics) claims that proponents of progressive immigration
policies are un-American as their advocacy supports the targeting of
their fellow citizens by immigrant violence.\footnote{Gage.} The
dominant poles in many such debates over aspects of Americanism have
indeed shifted unpredictably through no single person's prescriptive
mandate. However, savvy analysts \emph{can} sometimes foresee things
like negative economic indicators heralding a rise in xenophobia that
colors people's conceptions of American identity in relation to
immigration, and each swing in these conceptions has happened because
people prescribed them. As described above, Wilson actively sought to
manipulate dominant notions of American identity to his advantage.
Similarly, to arrive at a time when prominent figures such as Senator
Sanders call voter ID laws `un-American,' civil rights leaders first had
to advocate a vision of an America that was obligated to extend key
civil rights, including access to the ballot, to people of color.
Overall, Gallie's point that an essentially contested concept's twisting
path cannot in general be reliably predicted and is not subject to easy,
conscious manipulation by an individual user of the concept (or of a
word or phrase that references the concept) still stands, but I find his
original formulation slightly too strong.

By Gallie's own admission, his remaining criteria are of lesser
importance than these four. As I do not see him as offering much in the
way of cogent arguments for any of them as necessary conditions, and he
does little to clearly apply them to his own examples of essentially
contested concepts, I will bypass them here. I have found `Americanism'
to satisfy three necessary criteria (and one superfluous one) for an
essentially contested concept, but this may only suffice to indicate an
essentially contested nature if the satisfied criteria are also
(jointly) sufficient. A concept that meets the second and third criteria
must, it seems, be one of three kinds: essentially contested; resolvable
by some rigorous, objective (e.g. scientific) assessment; or simply
confused.\footnote{Here, I ignore elements of epistemological skepticism
  and underdetermination that most people leave out of the processes of
  making everyday judgements and ordering the world into conceptual
  schemes. As such, if a concept with these markers of complexity cannot
  be explicated by some objective process, a given party either does or
  does not employ the concept in ways that---while contradicted by
  opposing parties' uses---have sufficient internal consistency to allow
  us to say the party's members generally use it competently rather than
  confusedly. If there are fewer than two opposing parties this can be
  said of, the concept is simply confused (or is not subject to general
  disagreement in the first place); otherwise, it seems it is
  essentially contested.} A concept that meets these two criteria is one
defined by a complex array of components, which at first sight seem open
to many, potentially equally valid, systems of ranking and evaluation.
When we add the fourth criterion (with my caveats), making the concept's
content indefinitely subject through changing circumstances to
substantial, (mostly) unpredictable changes which are (mostly)
unsusceptible to deliberate prescription, I believe we eliminate both
objective and hopelessly confused concepts without the need for any
further criteria. Trivially, a concept's content cannot be objectively
and determinately settled if it is always open to unpredictable and
substantial change. Further, if it always admits of such change, it
seems it cannot simply be confused, as groups of users are able to agree
on meaningful shifts in use in response to shifting circumstances,
suggesting the presence of some theoretically definable diverging
standards which disputant parties employ competently. Therefore, I take
these criteria to be sufficient to a concept's being essentially
contested and find that they mark Americanism as such a concept. While
the gap between each major political party's conception of its own
members' views and its opponents' conceptions of those views may play a
role in their clashing applications of the `American' and `un-American'
labels, it seems the divergence begins from the essentially contested
nature of the concept itself. I will argue in the following section that
this does not, however, exhaustively explain the labels' clashing
applications.

\section{Further Disagreement and Epistemic
Uncertainty}\label{further-disagreement-and-epistemic-uncertainty}

We have seen some clear points of divergence in present usage of this
concept between the two parties, such as contemporary mainstream
conservative thought positing American values as Christian values yet
simultaneously subscribing to a binary that opposes `pro-immigrant' to
`pro-America'---while contemporary mainstream liberal thought opposes
both views by locating cultural pluralism at the center of Americanism.
Gage largely roots such oppositions in the contrast between contemporary
liberals' aspirational use of the concept (``pointing toward an American
dream of liberty and equality that has never quite been realized on the
ground'') and a more functional Trumpian view on which the concept is at
its core a tool for evoking national identity in a way that delineates
between a favored `us' and an undesirable `them.'\footnote{While I find
  this account to broadly capture how these groups presently use the
  concept, there remain clear examples wherein representatives of both
  sides attempt to root charges of un-Americanism in a
  \emph{descriptive} account of historical American identity, as seen in
  both of the Obamas' 2016 convention speeches and the constant evoking,
  across the political spectrum, of the nation's origins and Founders'
  values to define what is and is not American.} Yet there are surely
many commonalities as well---both parties' representatives would broadly
agree, for instance, that Americanism includes values of hard work,
service and sacrifice, freedom, democracy, courage, and entrepreneurship
(though they may disagree on how to define some of these values, some of
which themselves rise to the level of essentially contested concepts).
Yet even where their definitions overlap, they disagree on which party
exemplifies them, pointing to a further disagreement about the nature of
the parties themselves.

Hayes' statement exemplifies this state of affairs, in which each side
disagrees with the other about what its own positions and values
are---just as few conservatives would agree with the mainstream liberal
view that conservatism values the advancement of white, cisgender men
and the wealthy through the suppression of everyone else, it would be
hard to find a liberal to agree with Hayes' assessment that liberals
oppose hard work, achievement, and faith. This tension highlights the
unavoidable presence of uncertainty throughout the political order.
Sluga indicates that not only are political structures `essentially
complex systems'---i.e., systems whose vast numbers of constitutive
elements stand in widely varying relations to one another, whose
relations and the related elements themselves are subject to change
(including entry to, and exit from, the system), and whose great
complexity makes them unsurveyable---but they are more specifically
`hyper-complex systems.'\footnote{Sluga, Hans. \emph{Politics and the
  Search for the Common Good.} Cambridge: Cambridge University Press,
  2014. 238-239.} A hyper-complex system (HCS) is an essentially complex
one made up of human actors whose perspectives on the system containing
them, as well as their perspectives on their fellow actors'
perspectives, shape how the system operates. To attempt an objective
survey of such a system would require not only an already unperformable
survey of a vast, complex array of relevant material facts, but also an
inexecutable survey of its members' epistemically inaccessible views.
HCSs thus introduce a distinctly higher level of unsurveyability---they
are ``maximally unsurveyable.''\footnote{Sluga, 240.}\textsuperscript{,}
\footnote{America is not just a HCS, but is recursively so in the sense
  that it satisfies a modified version of the above HCS definition
  wherein the vast array of HCSs contained within America and their
  perspectives on one another replace human actors and their
  perspectives on one another, respectively, in the original definition
  (here, what I mean by ``HCSs' perspectives'' mirrors what I say about
  political parties' intentions in footnote 9). As such, America is
  undoubtedly maximally unsurveyable. Consequently, the more one's
  concept of Americanism appeals to a descriptive account of historical
  American identity over a merely prescribed set of values, the more
  egregious a bad-faith error one makes in boiling a maximally
  unsurveyable system down to a narrow, unifying character. Within this
  vast system, virtually anything one might attempt to class as American
  \emph{or} un-American has been seemingly both embraced and opposed at
  different times in different corners of American institutions.}

From their vantage points within a maximally unsurveyable system,
politicians cannot hope to achieve certain knowledge of the vast chain
of interconnected operations which animate it or their colleagues' views
that help shape these operations. Politicians' views may be made even
more inaccessible to one another (and the general public) by political
incentives to conceal or distort them.\footnote{Sluga, 237.} Further, we
are often drawn to interpret group actions as embodiments of a
collective will with all group members acting from the same motives to
aim at the same objectives, obscuring the varied and conflicting motives
and objectives of individual actors that more often characterize group
actions.\footnote{Sluga, 238.} Therefore, even with regard to elements
which both parties' conceptions of Americanism have in common, the
inaccessibility of political actors' (and their supporters') views
leaves each party's supporters in disagreement over which party's
members' views tend to better exemplify American values. For instance, a
large, complex bill containing some increase in federal spending on
welfare programs may be produced by a series of compromises between many
legislators---with no one involved aiming to produce, or even able to
foresee, exactly what is ultimately written into the bill---then pass
with most of its support coming from congressional Democrats. The
uncertainties surrounding the process and the individual actors' motives
may then leave room for conservative media outlets to attribute its
passage to Democrats sharing an un-American hatred of ``hardworking,
real Americans'' and a desire to ``make people dependent on the
government.''\footnote{The hyper-complex nature of welfare programs
  themselves also doubtlessly helps such conservative myths about
  welfare gain traction.}

These epistemic limitations constrain not only the actors and
institutions my sources and I analyze here but our ability to build any
certain analysis on stable, definite foundations. Our analysis of
various actors' views of Americanism is clouded by the uncertainties
catalogued above. Moreover, our positions within the cultural system we
attempt to analyze both limit the range of concepts from which we may
construct any analysis to those concepts made available by the system we
wish to analyze and confine this analysis within the invisible
boundaries of the system's internal logic.\footnote{Sluga, 240-241.} The
latter shaped the context in which we learned how to reason in the first
place, so this internal logic is for us epistemically inseparable from
what we perceive as natural or objective methods of reasoning.
Therefore, while I find that contemporary disagreements over who and
what is American or un-American are best explained by a combination of
Americanism's status as an essentially contested concept and the
uncertainty clouding disputant parties' assessments of each other's
character, the certainty of my conclusion is limited by its roots in an
analysis shaped by the very same conditions of uncertainty it attempts
to analyze.\footnote{There is much left to explore that would shed some
  more light on the question at the heart of this paper. This includes
  questions concerning the relation between concepts of Americanism and
  patriotism or nationalism, the relation between the uses described and
  American citizenship or residence, and how other countries' concepts
  of national identity align with and diverge from these. The balance of
  description and prescription in the concept's use could also be
  explored much more deeply. Finally, the reader may find that I have
  failed to adequately address my question as originally framed---I do
  not definitively say which competing characterizations are more
  central to the dispute, those of the political parties or those of
  Americanism itself. It seems to me that in resisting opponents'
  characterizations of them of the sort in section 1, liberals would
  largely emphasize conservatives' misportrayal of their views and
  actions while conservatives might lean more heavily on resisting the
  inclusive, pluralistic conception of `Americanism' they're accused of
  failing to meet. However, uncertainties of the sort Sluga highlights
  prevent me from reaching a full, satisfying answer to this question.}

\clearpage
\section*{References}
{
\small
\begin{itemize}[label={},itemindent=-2em,leftmargin=2em]	
	\item The Politico. ``Bachmann Calls for `Penetrating Expose' on `Anti-Americans' in Congress.'' \textit{CBS News}. 17 November 2008. 

	\item Stein, Sam. ``Palin Explains What Parts of Country not `Pro-America'.'' \textit{Huffington Post}, 17 November 2008. 
	
	\item Layton, Lindsey. ``Palin Apologizes for `Real America' Comments.'' \textit{Washington Post}, 22 October 2008. 
	
	\item Wetzler, Libby. ``Un-Amercan and Unclear: What Exactly Does it Mean to be Un-American?'' \textit{Fordham Political Review}, 27 April 2016. 
	
	\item Zorn, Eric. ``Column: No need to distort Trump's and Clinton's words. The truth is bad enough.'' \textit{Chicago Tribune}, 2 August 2016. 
	
	\item Washington Post Staff. ``Transcript: Read Michelle Obama's full speech from the 2016 DNC.'' \textit{Washington Post}, 26 July 2016. 
	
	\item \textit{ABC News}. ``FULL TEXT: President Barack Obama's 2016 Democratic National Convention Speech.'' 28 July 2016. 
	
	\item Gage, Beverly. ``How 'Un-American’ Became the Political Insult of the Moment.'' \textit{The New York Times Magazine}, 21 March 2017.

	\item Krammer, Arnold. \textit{Undue Process: The Untold Story of America's German Alien Internees}. Lanham, MD: Rowman \& Littlefield, 1997. 

	\item \textit{The New York Times}. `Gregory Defines Alien Regulations.'' 2 February 1918. 

	\item Gallie, W. B. ``Essentially Contested Concepts.'' Oxford University Press. \textit{Proceedings of the Aristotelian Society}, New Series, vol. 56 (1955-1956), pp. 167--198.

	\item Sluga, Hans. \textit{Politics and the Search for the Common Good}. Cambridge: Cambridge University Press, 2014.
\end{itemize}
}
