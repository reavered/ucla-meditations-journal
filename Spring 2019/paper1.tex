\section{Introduction.}

Aspects of contract law such as the consideration doctrine or efficient
breach\footnote{Consideration doctrine requires that there be some
  exchange between parties for a contract to be legally enforceable.
  Efficient breach allows for willful breach of contract in cases where
  a more profitable option comes along, and requires compensation to the
  promisee (which is offset by the greater profit from the new
  opportunity).} reflect a divergence of contract law from the
conventions of promising. Promise theorists, such as Seana Valentine
Shiffrin, argue that these instances of divergence need to be reconciled
or justified in the face of moral norms of promising that they go
against. Other legal scholars, such as Michael Pratt, argue that we need
to rethink the common conception of contracts as promises. If contracts
can be understood as a kind of commitment distinct from promising, then
no justification for divergent doctrines of contract law is needed. This
paper will navigate the debate between Shiffrin and Pratt regarding an
example Pratt offers that purports to be a contract devoid of promissory
moral norms.The debate centers on the descriptive claims of promise
theorists, and I will aim to follow that, setting aside prescriptive
concerns.

I will begin by explaining Pratt's hypothetical contract which is not a
promise, and his reasons why it is implausible that a voluntary legal
commitment is also necessarily a moral commitment. I will follow this by
examining Shiffrin's two lines of analysis of Pratt's claims. The
exchange as we will then have it seems unresolved, although, there is a
question that Shiffrin sets aside that I think will help to adjudicate
this debate: Are there commitments between people that do not bear the
moral obligations of promises? Both Shiffrin and Pratt agree that there
is a commitment taking place in the hypothetical contract that is not a
promise.\footnote{This paper will primarily be engaging with three
  texts: Shiffrin ``Are Contracts Promises?'' (ACP), Pratt ``Contract:
  Not Promise'', and Shiffrin ``The Divergence of Contract and Promise''
  (DCP).}\footnote{(Pratt 801, Shiffrin ACP 17{[}2{]})} I will argue
that it is not possible to successfully make a commitment to a person
without taking on a promisory moral obligation. Making a commitment to a
person simultaneously gives that person a moral claim to your conduct
meeting that obligation. I will also consider how Pratt's hypothetical
resembles other speech acts that rely on constrained definitions of the
concept of which they are about. Along the way I consider objections
that Pratt and Shiffrin might level against my conception of commitments
as promises and offer replies to those objections. Brief concluding
remarks will close the paper, though likely not the debate.

\section{Pratt's argument for a non-promissory kind of
commitment}

Pratt wants to engage with promise theorists on their descriptive claims
about contracts but not on their prescriptive claims about contract law.
Pratt claims the view that contracts are promises is too often reduced
to orthodoxy and mere stipulation by legal scholars.\footnote{Pratt,
  801.} Pratt does not think that contract law deals exclusively with
promises, but rather agreements that are sometimes promises.\footnote{Ibid,
  802.} If there are some contracts that are wholly not promises, then
contract law cannot rightly be described as dealing only with promises.
This would in turn mean that contract law's purpose would only have a
coincidental and not instrumental relationship to enforcing
promises.\footnote{Ibid, 816.}

In order to defend his claim that some contracts are not promises, Pratt
offers the following hypothetical example. Our characters are Eliza the
homeowner and Rudy the electrician. Eliza has an electrical problem in
her home that a previously contracted electrician failed to locate and
repair. To avoid a recurrence of the problem, Eliza asks Rudy to promise
to repair the electrical problem prior to starting the work. Rudy is
confident in his work, but reticent to give others moral claims over
himself. He offers to fully guarantee his work by contract, but disavows
any moral claim or obligation that might otherwise arise with his
commitment to do the work. Eliza finds this request unobjectionable, and
agrees to enter into contract with Rudy for the electrical work.

This situation, or something very close to it, would more than likely
amount to a legal contract. But is it a commitment that successfully
departs from anything having to do with a promise? Pratt seems to accept
Shiffrin's definition of a promise as a voluntary commitment to a course
of conduct ``to which certain definite moral norms attach, including a
requirement that they be performed.''\footnote{Ibid, 809, 810. Pratt
  uses a language of ``undertaking'' in reference to Rudy's contract,
  and ``course of conduct'' in his definition of promise. Pratt's
  definition does more to emphasize promising as a voluntary speech act.}
So, we know at least one of the certain definite moral norms that are
part of a promise: a requirement to perform. Pratt defends this narrow
conception of promising by contrasting it with the definition offered in
the Restatement (Second) of Contracts,\footnote{The Restatement is an
  academic document by legal scholars that summarizes contract law and
  is often referred to by courts, but is itself not a legal document.}
which he seems to think is too broad to capture anything of
``substantive morality.''\footnote{Ibid.} The Restatement definition
hinges on one person manifesting an intention to another person such
that the other person is justified ``in understanding that a commitment
has been made.''\footnote{Restatement (Second) of Contracts § 2 (1981).}
Pratt thinks that signaling to another person that one is undertaking a
commitment does not capture the important aspects of what we typically
consider a promise. Getting past semantics, the instrumental quality of
a promise is that it gives rise to a moral debt of performance to
another person, and the Restatement definition says only that one acted
in such a way as to justify the promisee's belief about the commitment.

Maybe it is the case that Rudy has not given a moral claim to Eliza for
conduct that is specifically- locating and fixing the electrical
problem. Pratt acknowledges that other moral obligations might attach to
Rudy's commitment, just not ones specifically characteristic of
promises, especially not the moral requirement to perform the repair.

What are these non-promissory moral obligations that Rudy may be subject
to because of this contract that is not a promise? Pratt suggests that
Rudy may be morally obligated to not allow Eliza to form a belief that
he will actually complete the repairs, because he has only ``undertaken
to complete them.''\footnote{Pratt, 809.} Wait a second. What is the
difference between committing to do something, and undertaking to do
something? Pratt says that if Rudy decides not to complete the job, he
may be morally obligated to inform Eliza of his decision in a timely
manner. Undertaking to complete a job cannot merely be a ``I might do
this, so these are the terms under which I might do it'' kind of
contract. Is that even a contract? The moral obligations that Pratt
acknowledges could attach to Rudy's non-promissory contract seem to bear
striking resemblance to the kind of paltry moral obligations we might
expect to fall under if we tell a friend that we will meet them for
dinner if our schedule frees up sufficiently. But this is less a
commitment to do something than it is the conditions under which we
might commit to do something. Rudy said he would guarantee his work by
contract. For now, I am going to set aside my concerns about what moral
obligations Pratt says might attach to Rudy's contracting, because Pratt
characterizes these as non-promissory. I want to move forward with the
assumption that Rudy is contracting to do the work, and try understand
what kind of a commitment this contracting might constitute.

Pratt's primary claim in support of the hypothetical contract that is
not a promise is called the voluntariness thesis. He says that to
conceive of Rudy's contract as a type of promise that adheres certain
moral norms but leaves him otherwise free to not perform the work shows
an inconsistency with the underlying voluntary conception of promising.
The underlying conception of promising is that promissory obligations
are a special kind of moral obligation that are voluntarily
self-imposed. Pratt wants to argue that morality has nothing to say
about undertakings ``simpliciter'' but only promissory undertakings.
This language of ``undertakings'' obscures the fact that these are not
just undertakings to do a thing, they are commitments. I can undertake
to walk down the street. If I agree to go for a walk with my friend, I
have both undertaken to walk somewhere and committed to my friend that I
will walk somewhere with them. There are various dictionary definitions
of the word undertaking, but I think it is safe to assume Pratt means
the word in the sense that it means a commitment to another of
performance of some task. So how can Rudy both undertake to do the
repair work and be free to choose not to do the work? Because, according
to Pratt, only moral undertakings actually require one to do what one
has committed to doing, and moral undertakings must, by the voluntary
nature of promising, be voluntarily accepted by the contractor. On
Pratt's view, it must be that mechanisms other than morality obligate a
contractor to some set of conduct. As for ``moral obligations of the
promissory variety'' holding a contractor to a course of conduct, that
can only happen if the contractor manifests an ``intention of creating a
moral obligation.''\footnote{Pratt, 812.} This is Pratt's voluntariness
thesis of promises.

The opposite of this he calls the no-disclaimer thesis; a moral
obligation of the promissory variety adheres if the contractor does not
disavow a moral obligation. Pratt says the no-disclaimer thesis is
implausible, because of the very reason the voluntariness thesis is
plausible. There has to be an intention to assume a moral obligation,
which may or may not be present despite the lack of disclaimer. If no
such intention is present and there is no disclaimer, there is no moral
obligation. The voluntariness thesis tells us more than enough about
cases where a disclaimer such as Rudy's is present; no promissory moral
obligation can adhere. In fact, Pratt claims that ``the voluntariness
thesis must be true because it provides the only plausible account of
why Rudy is not morally obligated by his undertaking.''\footnote{Ibid.}
So, Rudy succeeds in making a non-promissory commitment because of the
voluntariness thesis of promising, and the voluntariness thesis of
promising is true because Rudy succeeds in making a commitment that is
not a promise.

\section{Shiffrin's response to the counter example.}

Shiffrin's conception of promising is similar to Pratt's narrow
definition. The two agree, fundamentally, that there is a moral
requirement to perform what is promised; though Shiffrin further
qualifies her own conception, claiming that, in addition to the moral
requirement, promising entails a transfer of power r to the promisee
that allows them to demand or excuse performance. Shiffrin's concern for
the ways contract law diverges from promising is what gave rise to
Pratt's article, and her attitude towards this divergence plays some
role in her analysis of Pratt's counter example. The legal consequences
for breach of contract are quite different than the moral consequences
for breach of promise, particularly in terms of disapprobation; the law
effectively offers none whereas disapprobation is the primary
consequence morality lays on breach. This might be one consideration in
favor of the view that promises and contracts are of a different nature.
However, this is only one facet of the comparison between contracts and
promises, and there are ways to construe this difference as concerning
the difference between law and morality rather than contract and
promise. But we are concerned with whether contracts as commitments are
of a promissory nature.(This seems to be a natural place to start a new
paragraph. Prior to this line, you discuss Shiffrin's conception of
promising and the intricacies it points out in the overall debate. Now,
you are shifting to her address of the counterexample.) Shiffrin
addresses the counter example in two ways. First, she suggests that Rudy
might be making a kind of linguistic mistake in that he is not
disavowing promissory norms altogether, just stipulating their scope and
strength. Second, she considers how the moral content of Rudy's contract
might still resemble promissory moral content.

Shiffrin's first line of analysis says that Rudy succeeds in promising
to either perform or pay, but that he fails to avoid promising
altogether. Eliza asks him to fix the problem. Rudy responds by saying
he will undertake the contract to fix the problem, but not promise to
fix the problem. We can draw from this that he is making a promissory
commitment to a broader set of actions that would satisfy
``performance.'' In this sense, ``perform or pay'' would be the promised
performance. Shiffrin offers some other examples of alternate promises
that we could ascribe to Rudy,\footnote{Shiffrin offers an alternative
  counter example that is similar to Pratt's but involves two friends,
  one contracting the other to build a bookshelf. From this she suggests
  that Rudy's promise may be as a promise between strangers instead of a
  promise between friends, which would carry heavier moral obligation
  than a promise between strangers. Much of her response revolves around
  her example of the counter example. I am primarily concerned with some
  aspects of her second line of analysis regarding what potential for
  moral wrong remains in Rudy's contract despite the disavowal of
  promise, and the question she leaves open at the end of her analysis.}
aside from a strict promise to do the work. She concludes that there is
no way to resolve the linguistic interpretation of the scenario that
remains charitable to both Rudy and to Contract Law's definition as
dealing with legally enforced promises.

The second line of Shiffrin's analysis looks at where we might correctly
say that Rudy behaves immorally, and whether such potential immoral
behavior would be the product of breaking a promise or shirking a legal
duty. Shiffrin considers what Rudy is committed to doing by contract,
such as perform or pay, and asks whether a failure to do either is a
moral wrong. She suggests that if Rudy were to refuse to either perform
or pay, and rather told Eliza to take him to court if she wants a remedy
for his breach of contract, that his action would seem to constitute a
moral wrong akin to breaking a minimally moral disjunctive
promise.\footnote{Shiffrin, 14.} She suggests that Pratt might argue the
moral wrong does not stem from breaking a promise that was not made, but
rather from Rudy's attitude towards the law or from the ``force of the
expectations {[}Rudy{]} cultivated in {[}Eliza{]} at the time of
contracting.''\footnote{Ibid, 15.} Shiffrin's notion of a promisor
cultivating expectations in a promisee is similar to the Restatement's
notion of a promisor justifying a promisee's understanding that a
commitment has been made. It is unclear why Shiffrin would allow this as
a non-promissory explanation of the resulting moral wrong, except that
her definition of promise is more narrow than the Restatement's.

Ultimately Shiffrin leaves the hypothetical aside. She thinks it is
``strange'' and maybe even ``oxymoronic'' to believe that a contract can
be completely untinged by the norms of promising.\footnote{Ibid, 9, 15.}
Accordingly, Shiffrin shifts gears to look at what is ``normatively
appropriate or preferable'' in answering the question concerning the
potentially promissory nature of contracts. In doing so she assumes
Pratt's position, that it is possible to have a system of contract that
is wholly separate from promising, and sets aside the ``valid concern''
as to whether there are commitments that successfully disavow promissory
moral obligations. I intend to pick up this question My argument expands
on both lines of analysis that Shiffrin offers, though I am more
concerned with the legal aspects of promising. Importantly, my argument
employs a broader conception of promises.

\section{Boundaries of conduct and promissory moral
failings.}

I want to take a step back and look at moral obligations and
commitments. We have a moral obligation not to harm other people, and
other people have a moral claim to respect for their bodily autonomy.
These and other moral obligations exist independent of whether people
choose them as obligations for themselves. In the broadest sense, moral
obligations say that some behavior is right, and some behavior is wrong,
on strictly moral and not legal grounds. Some behavior is both morally
and legally determined to be right or wrong, but typically such behavior
bears either quality independent of the other. Promissory moral
obligations arise from a set of voluntary behaviors, and so Pratt argues
that promissory moral obligations are voluntarily assumed. However, I
find it more plausible that one voluntarily makes a promise, but
promissory moral obligations necessarily follow from the act of
promising. what is voluntary is the making of a promise, and the
promissory moral obligations are inalienable from making a promise. To
bring this out, we need to look at what is involved in making a promise.

It is not my intention to define promises, but simply to consider some
aspects of making a promise. I think Shiffrin and Pratt would agree that
promises involve commitments and that contracts involve commitments.
That commitments involve inalienable promissory moral obligations will
be much more challenging to bring everyone into agreement on. That is
what I aim to do.

In order for a contract to be legally binding there needs to be an
offer, an acceptance, and a consideration.\footnote{My discussion of
  contracts and commitments will look as if there is no consideration,
  because I intend to discuss the positions of contractor and
  contractee. I think my discussion can account for contracts that have
  consideration by simply mirroring the dialectic between the two
  parties.} An offer is a proposal by one party to commit to a course of
conduct, and acceptance is the other parties accepting claim to that
conduct by the offering party. If a party commits to a course of conduct
that includes all possible courses of action, then they have not
actually committed to anything. If I ask my friend if they would like to
go to dinner and they answer ``yes, but I might change my mind at any
moment and I won't be able to let you know if I change my mind,'' then
my friend hasn't really made a commitment.. Maybe I can place faith in
the belief that their desire to go to dinner with me is strong enough
that they will show up, but that does not constitute a commitment on
their part. . A commitment, establishes a boundary between two sets of
possible behavior or conduct on the part of the committed person. One of
these sets of possible conduct will meet the commitment, and the other
set of possible conduct will fail to meet the commitment. Sometimes the
set of possible conduct that will meet the commitment will be very
small. In Shiffrin and Pratt's conceptions of promising, we might say
that the set of conduct that meets a promissory commitment is the
performance of the ``undertaking.'' If I promise to go to dinner with
you, I am committing to a very narrow course of conduct. If I promise my
mother that I will care for and look after my siblings for the rest of
my life, or even the rest of the weekend, I am committing to broader
course of conduct.

So, we have on the table the idea that commitments give rise to possible
actions that meet the commitment and possible actions that do not meet
the commitment\footnote{This argument could be seen as an expansion on
  Shiffrin's suggestion that one might insist that Rudy ``promised
  \textit{something}'' (ACP 13). What I have set out to do is suggest that
  this something is whatever the content of the commitment is.}. Is it
possible to go further and say that there is an inalienable moral
obligation to behave or conduct oneself within the set of behaviors that
meet the commitment? I think that, in any case in which an actual
commitment is made, the committed person is morally obligated to fulfill
that commitment, barring conditions for excuse and/or remedy of failure.
One might reply that commitments can be made and then dropped. Well,
barring excusatory reasons,this doesn't seem to be a true commitment. If
there are excusatory reasons for dropping the commitment, then the
previously committed person's behavior falls into a category of conduct
that is penumbral to the conduct that would meet the commitment. It
could even be thought of as conduct that would otherwise have met the
commitment, if not for the valid excuse. Conditions for excuse from a
commitment might be copious or scarce, but gauging the scope of
conditions for excuse is going to be highly context sensitive. And
sometimes failure to perform will be excused upon some kind of remedy,
and sometimes failure to perform will be excused with no remedy for the
failure to meet the commitment. Maybe there are some commitments that
require performance, some that require performance or excuse, and still
others that require performance---or excuse, or remedy. It is not at all
my goal to say specifically and definitively what commitments or
promises require, just that they do create a set of requirements on the
promisor for the commitment to be fulfilled (or excused or remedied).

The question then becomes, who says fulfilling a commitment is right or
wrong? This might be the biggest move of my argument. The law can fully
usurp the role of morality in setting the boundaries between conduct
that does and does not meet meets the commitment---but as long as a
commitment has been made, the person it is made to has a moral claim for
performance, excuse, or remedy. There is an inalienable moral obligation
attached to any commitment to another person.\footnote{And maybe even to
  commitments made to one's self, though that is quite beside the point
  of this paper.} There seems to be a notion floating around that
promissory moral obligations only arise when morality plays a role in
defining what meets a commitment, i.e., very specific performance is
demanded, and a moral failing occurs when specific performance is not
met. So, Pratt comes to the conclusion that if specific performance is
not demanded,, then there was no moral obligation attached to his
commitment to begin with. This is objectionable. Either Rudy committed
to something other than specifically performing the work (which might
come as a surprise to Eliza), or he did not actually commit to anything
at all. Assuming that Pratt conceived of the counter example as a
plausible example of a contract that the law would enforce, then there
must be something that Rudy is offering up, some commitment that he is
offering.

If we look back at the non-promissory obligations that Pratt says attach
to Rudy's commitment, it might look as if Rudy failed to make a
commitment, or at most he committed to either doing the work or
informing Eliza that he changed his mind (Pratt 809). I suspect that
Pratt's response would be that a contractual commitment only gives rise
to contractual obligations and not (necessarily) moral
obligations.\footnote{Pratt, 807.} If this is the case, there must be a
set of behaviors that the contract stipulates as meeting the commitment,
and a set of behaviors that do not meet the commitment, some of either
of which might be filled in by the default rules of contract law. If
Rudy totally absconds from any conduct that is stipulated as meeting the
commitment, is he really morally unaccountable? This seems very
unlikely. Suppose that morality lays no judgement on Rudy's conduct
regarding the contractual commitment. Rudy offered to ``fully guarantee
his work by contract''\footnote{Ibid.} in response to Eliza's request
that he promise to locate and fix the problem. Suppose Rudy was
confident in his electrician skills, and fully intended and believed
himself capable of locating and fixing the problem, thus he offered a
genuine guarantee of the work through contract. But the contract holds
no promissory moral water regarding specific performance. Suppose Rudy
gets under the house and it is such an unpleasant experience, he decides
to finish the job. Trying to pass off his efforts as meeting the
guarantee falls into a different category of moral failing, namely lying
or fraud. But if he simply comes out from under the house, and without
any form of apology for failing to fulfill the commitment, packs up and
leaves, it would seem like Rudy fell short of some kind of demand of
morality, not just short of his contractual obligation. If a contractual
obligation is an agreement to a set of behaviors that either party can
opt out of at any time, then it is not the case just that contracts are
not promises, it would also be the case that contracts are not
commitments.

There are actions that sometimes look like commitments but are in effect
just a suggestion that one has an inclination towards a certain course
of action. If I say that I will meet you for dinner if I am not
otherwise enticed by Netflix, I have clearly not made a very reliable
commitment. Signaling an intention is not the same as making a
commitment, even though they sometimes look similar. At most, signaling
an intention might be a host of disjunctive commitments,\footnote{Shiffrin
  suggests a similar analysis of Pratt's hypothetical contract as ``{[}a
  contract{]} with unusual, disjunctive terms''(ACP 14). The set of
  conduct that would fulfill a contractual commitment might in some
  cases have disjunctive elements, but there would at least be a set of
  conduct committed to. In some cases specific performance of a promise
  is simpling picking out a course from a disjunctive set of conduct.
  The prescriptive claims of promise theorists should do the job a
  narrowing the set of conduct. The descriptive claims cannot do enough
  work to show that any contract should have a non-disjunctive set of
  conduct by which it is fulfilled.} and the moral failing would occur
if a course of action was taken that had no concern for any of the
disjunctive commitments (and someone had a reasonable expectation that
you would to live up to your word of doing X, Y, or Z). But the act of
signaling intentions likely never amounts to much of a contract, and is
probably the weakest kind of promise, if it amounts to either at all.

Pratt might get on board with my connection of contractual commitments
to promissory moral obligations if it is clear that contractual
commitments are only commitments to the terms of the contract and
whatever gap filler default rules apply\footnote{Default rules of
  contract law cover aspects and terms of a contract that the
  contractors do not themselves stipulate, when those aspects or terms
  need to be stipulated.}. A moral failing occurs on the part of the
contractor when they abscond from any conduct the contract or contract
law stipulate as meeting the contractual commitment. Maybe it is such
that non-contractual promissory commitments place morality in the role
of the contract and contract law, and morality places stricter
boundaries on the conduct that would meet the commitment or merit
excuse. But in either case there is a commitment that stipulates conduct
that would meet that commitment, and failure to meet that commitment is
a moral wrong, on top of whatever legal or non-legal consequences the
failure entails. Producing the conditions that justify another person's
belief that a commitment has been made gives rise to the potential for
conduct on the behalf of the committed person that is either morally
right or wrong. One objection to this conception of the connection
between contracts, promises, and commitments is that moral failings only
occur when morality plays a role in stipulating the conduct or excuses
for the commitment. There is more to explore in this line of objection,
but I think that in as much as not meeting a commitment that another
person had a justified belief that you would meet is wrong, it is also a
failure to meet a moral obligation of the promissory variety.

I suspect that Shiffrin would disagree with my separating the scope of
conduct that meets contractual commitments and non-contractual
promissory commitments, while still claiming the two bear enough in
common to say that contracts are promises. To address this concern, I
would like to look for the plausible moral failing in a housing lease
contract. Imagine I have a contract to rent a room for a year, and it is
stipulated in the contract that if should I move out early, then I am
responsible for the remainder of the terms rent. Suppose I move out
after 6 months, not due to any problems in the rental. The landlord
meets her responsibility to mitigate loss, but there is still 6 months
rent that I am contractually obligated to pay. If I pay the remaining
rent, I have fulfilled the stipulated conduct to meet my commitment. If
I skip town and force the landlord to send the bill to collections, I
would seem to have failed to meet a moral obligation that arose from my
contractual commitment. While it might be kind of a stretch, maybe
economic considerations like efficient breach could be thought of as
default rules akin to the rental contract that states I must pay for the
term of the lease. In this case, efficient breach would be part of the
set of conduct that meets the contractual commitment. Ultimately, I am
more willing to accept that conduct that meets contractual commitments
is a broader set of conduct than Shiffrin believes it should be in some
cases. It does not seem plausible that all promises have only have only
one item in their set of conduct that would fulfill the promissory
commitment. Even so, we could view the set itself as one item, and
specific performance as conduct that is contained by and stipulated in
that set. I think that there is room to argue that contract law should
reign in the conduct it considers as meeting certain contractual
commitments, while maintaining my overall architecture that links
contracts to commitments to promises.

\section{Disingenuous narrowing of definitions.}

I would like to subject Pratt's hypothetical, non-promissory contract to
one more line of analysis. This is not intended as accusatory or
denigrating to Pratt's philosophical reasoning, but it seems like a very
salient line of comparison.\footnote{I hope that this very sentence
  doesn't come off as purporting to not be the very thing that it is.}
There are many examples in society and culture of things that claim to
not be something or have some quality, but they are very much that thing
or have that quality. I will run through a few examples of these and try
to identify what they have in common with Pratt's counter example.

If I were to be auditing a class, but tell others I am not a student,
there would be some sense in which this is true and some sense in which
this statement that I am not a student is false. Is a student a person
who studies certain subjects and seeks to grow in their knowledge of
those subjects, or is a student someone who is enrolled in an
institution that identifies its enrollees as students? Well, when I go
up to the counter at the movie theater and ask for a student discount,
the only relevant definition will be the one that hinges on an
institution that will issue me an ID that verifies me as a student. In
some contexts, the narrower definition of student is important to our
society, but that doesn't negate the broader definition of student. In
fact, the broad definition of student might not be captured by some
instances that are captured by the institutional definition of student.
Pratt's definition of promise as ``including a requirement that the
{[}committed undertaking{]} be performed''\footnote{Pratt, 809.} might
be important to some instances of promise, but it does not necessarily
negate all the other instances of commitments that are promises (and
sometimes contracts).

Another example comes from my older brother many years ago. He once had
a bong that had a sticker on it that said ``Not-a-bong''. A bong is a
water pipe that is typically used to smoke marijuana. 20 years ago, my
older brother was in his late teens and marijuana was still quite
illegal. Headshops, the shops that sold marijuana paraphernalia, would
tend to get upset with customers who walked in off the street and asked
to see the bongs they had for sale. Smoke shops and headshops could sell
water pipes that were used for smoking tobacco, but it was illegal to
sell bongs that were used for smoking marijuana. This is an example of a
knowingly disingenuous narrowing of the definition of a thing. It served
the headshops purpose to say that this thing is a water pipe but not a
bong, when in fact it was very much both. Similarly, to say that
contracts are commitments but not promises may serve a purpose helping
to deny the need for moral justification of the divergence of contract
law from promising, but the making of contractual commitments still seem
to serve the same kind of purpose as promises.

Also, we have the ``I am not a racist, but\ldots{}'' example. Often what
follows the ``but'' is a stereotype or generalization of an entire race
or culture of people. Some people who engage in this trope genuinely
believe that racism is only narrowly defined as conscious and
intentional hate for an entire race of people. In fact, there are many
manifestations of racism, some of which are not conscious at all, and
most if not all of which, have to do with stereotyping or generalizing
an entire race of people. A strict and narrow definition of racism has
hundreds of thousands of people convinced that their racist behavior and
comments are not in fact racist.

If we accept that promises are only those commitments that intend to
give a moral obligation to a very specific action, then we are missing
the substantive moral content of the vast majority of commitments. Pratt
would only agree ``under some widely accepted definition of that
term''\footnote{Pratt, 809.} that Rudy's contract is a promise. But what
is important to Pratt is whether Rudy is committed to doing the
repair---this is the specific question of ``substantive morality'' that
Pratt wants to answer about Rudy's contract as a promise. It is not the
case that Rudy's contract makes him in debt to Eliza, morally or
legally, to perform the repairs. Therefore, Rudy's contract with Eliza
does not amount to a promise to perform the repairs. The problem with
this has already been illustrated. If the contract is any kind of
commitment, then it amounts to a promise to meet that commitment with
something from a set of conduct. Promises can often have more than one
way of being fulfilled. And a promise that is not to do one thing does
not mean that it is not a promise for other things. And whatever the set
of behaviors turn out to be that would fulfill Rudy's contractual
commitment to Eliza, the requirement to engage in some or all of those
behaviors to fulfill his commitment to Eliza is the inalienable
substantive moral content of his contractual commitment. To do otherwise
would be a moral failing, barring excusatory reasons for failure to meet
the commitment.

\section{Conclusion}

Shiffrin claims that there is no clear consensus on what constitutes a
promise.\footnote{Shiffrin, 6.} I have tried to avoid saying too much
about what a promise is, other than to say that a promise is a
commitment that sets out morally right conduct that meets that
commitment and morally wrong conduct that fails to meet that commitment.
Whatever else a promise is, and however we agree to stipulate what
conduct meets what commitments, promises and commitments seem to be
intrinsically and instrumentally connected by these qualities. Pratt's
hypothetical contract that appears non-promissory either fails to be a
commitment, or is in fact a promise.

\clearpage
\section*{References}
{
\small
\begin{itemize}[label={},itemindent=-2em,leftmargin=2em]	
	\item Pratt, Michael G. ``Contract: Not Promise.'' 

	\item Shiffrin, Seana V. ``The Divergence of Contract and Promise.'' \textit{Harvard Law Review}, vol. 120, no. 3, 2007, pp. 708--753.
	
	\item Shiffrin, Seana V. ``Are Contracts Promises? (pre-publication version).'' 
	
	{\footnotesize \texttt{www.law.ucla.edu/~/media/Assets/Law\%20and\%20Philosophy/Documents/Shiffrin -Contracts-Promises.ashx}}
\end{itemize}
}