This paper examines Roger White's Bayesian objection to perceptual
Dogmatism and defenses on behalf of Dogmatism. Perceptual Dogmatism is a
promising view that directly works against certain external world
Skepticism and can be used as a form of modest foundationalism to block
the regress argument for justified belief. The view holds that
perceptual experience alone can provide immediate justification for a
belief without basing epistemically on other justified beliefs. The
Bayesian objection to Dogmatism comes from Bayesian epistemology, the
field that concerns with the rationality of the credence of, or degrees
of confidence in, beliefs. The objection shows that if Bayesian modeling
is used for Dogmatism scenario, then the results go against what
Dogmatism expects. So, Bayesianism and Dogmatism cannot be jointly
consistent. I will present two replies on behalf of dogmatism to the
Bayesian objection. Through evaluating these two replies, I come to the
conclusion that the problem of Bayesian objection lies in the
application of Bayesian model to precisely track the concepts of
Dogmatism.

Throughout the paper, I use a scenario where an agent perceives a red
ball to illustrate the details of each view. I abbreviate the agent's
experience and relevant beliefs as the following:
\begin{itemize}
	\item[--] Experience E: perception as if there is a red ball

	\item[--] Belief I: I have a perception as if there is a red ball

	\item[--] Belief R: there is a red ball

	\item[--] Belief F: I am deceived to think there is a red ball when there is not

	\item[--] Belief \(\neg F\): I am not deceived to think there is a red ball when
there is not
\end{itemize}


\section{Pryor's Dogmatism}

I want to first present the precise formulation of perceptual Dogmatism
by James Pryor in "The Skeptics and the Dogmatists" (2000). Pryor
situates dogmatism as a response to the radical Skepticism that
perceptual experiences cannot give any justification for belief about
the external world. Pryor formulates the response by rejecting the
Skeptic's premises. The premise that will be relevant to the paper is
the following: "{[}SPJ{]} If you're to have justification for believing
\(P\) on the basis of certain experiences {[}...{]} {[}e{]}, then for
every \(Q\) which is 'bad' relative to {[}e{]} and \(P\), you have to
have antecedent justification for believing \(Q\) to be false" (Pryor
531). \(Q\) is considered as \emph{bad} relative to e and \(P\) if it
retains certain special feature of skeptical scenario such that
experience e would still be obtained even if \(Q\) is true, and that
\(Q\) is incompatible with \(P\). In the above scenario I set up, F is
\emph{bad} relative to E and R, because one can still have perception as
if there is a red ball even if he/she is being deceived to perceive so,
and F is logically incompatible with R. Furthermore, justification for
\(P_{1}\) is \emph{antecedent} to \(P_{2}\) if and only if the reasons
that justify \(P_{2}\) \emph{are} not presupposed as part of the
justificatory source for \(P_{1}\) (525). So SPJ requires that
$\neg F$\footnote{The symbol "\textasciitilde{}" is used as
  "not". So $\neg F$ means not-F.} has to be justified in a
way that does not use any justification for R in order for R to be
justified.

Dogmatism refutes SPJ by holding that one gains immediate prima facie
justification for \(P\) in virtue of having a perceptual experience as
of \(P\). Thus, one does not need antecedent justification for the
falsity of skeptical scenarios that \(Q\) represents in order to believe
the content of perceptual experience \(P\). Note that having experience
as of \(P\) does not entail that one needs to be aware of it or using it
as evidence to arrive to \(P\). \(P\) is \emph{immediately} justified
for an agent if and only if its justification does not rely on\footnote{Note
  that the "rely on" and "dependence" (used later) is only about
  epistemic or justificatory dependence, rather than psychological or
  physical dependence.} any evidence or justifications for other
propositions (546). The immediacy requirement focuses on the basis of
justification and asks whether such justification is dependent on other
justifications or is obtained independently. \emph{Prima facie}
justification for \(P\) is justification whose strength can be weakened
in face of additional evidence. The evidence that weakens the strength
of justification for \(P\) is called defeating evidence, or defeater.
Defeating evidence can be undermining or overriding. Undermining
evidence defeats the validity of the original justification, while
overriding evidence provides positive account for
\textasciitilde{}\(P\)\emph{.} For an agent, without having such
defeating evidence, prima facie justification becomes all things
considered justification. All things considered justification is the
final justification at a time, taking into considerations of the all
relevant evidences the agent have at the time. All things considered
justification needs not to rule out the existence of defeaters (that
could but not yet acquired) but only need to outweigh the defeating
evidence already acquired (545). Finally, the kind of propositions that
that can be justified as such is what Pryor calls perceptually basic
propositions. Perceptually basic propositions represent the contents of
perceptual experience, or the content that are directly delivered by
perceptual experiences\footnote{As the propositional attitudes are
  different for perceptions and beliefs, propositional contents are
  different. Here by representation, I assume that the contents can be
  isomorphically transformed to each other.}. In the particular case set
up, if one arrives to R from E, then R is prima facie justified. If the
agent is not presented with other evidences that would defeat R, then R
is an all things considered justified belief. $\neg F$ needs
not to be justified antecedently to R and R can be undermined by
defeating evidences.

\section{White's objection}

Before explaining White's Bayesian objection, I want to lay out some
basic ideas and principles of Bayesianism that White employs in his
argument and will be useful for later discussion in evaluating his
objection and others' reply.

The first idea is probability function. For an agent, the probability
function P takes in a proposition and outputs a number between 0 and 1
that represents the degrees of confidence that the agent has in the
proposition. Such degrees of confidence of a belief is also called the
credence of a belief. Note that an agent needs not to believe the
proposition A for the agent's degree of confidence for A to be
represented by \(P(A)\). Credences of a proposition A are given under
full belief states and partial belief states. In full belief states, the
credence just is \(P(A)\). While in partial belief states, the credence
of A is given as the conditional probability of A on B, defined as
\[ P(A|B) = \frac{P(A \cap B)}{P(B)}.\]
Then, I want to introduce the idea of rationality of the credences and a
set of principles for rationality. For an agent's credences to be
rational, the credences of all propositions the agent believes should be
consistent with each other under the probability laws. For proposition A
that is not believed by the agent, with the rational constraint on
credence, \(P(A)\) represents the rational belief attitude towards A
given the epistemic state the agent is in, were the agent to take any
attitude towards it. The probability laws include:
	\begin{enumerate}[label=\arabic*.]
		\item for any proposition A, \(0 \leq P(A) \leq 1\)\footnote{Corollary: all
  conditional probability is larger or equal to 0 and is smaller or
  equal to 1.}.
  		\item if A is a tautology, then \(P(A) = 1\). 
  		\item if two propositions (A and B) are disjoint, i.e. they cannot be both true, then
\(P(A \vee B) = P(A) + P(B)\). 
		\item The conditional probability of A on B
can be derived from the conditional probability of B on A, following
Bayes' theorem 
			\[ P(A|B) = \frac{P(B|A)P(A)}{P(B)}. \]
	\end{enumerate}		
Such rules are
usually considered as synchronic constraints on the credences, meaning
that constrains to the distribution of credences of propositions at a
particular time segment. Aside from synchronic constrain by probability
law, conditionalization principle gives the diachronic constrain on
credences. In particular, it concerns the rational change in credence
for an agent after him/her gaining new evidence. The credence for
proposition A before gaining the new evidence is the initial credence,
represented as \(P_{i}(A)\); the credence after gaining the new evidence
is the final credence, represented as \(P_{f}(A)\). Gaining new evidence
is represented as becoming certain of the proposition B that represents
the evidence, i.e. \(P_{f}(B) = 1\). Conditionalization principle
requires that after gaining evidence, \(P_{i}(A)\) should be updated to
\(P_{f}(A)\) which equals to the conditional probability of A on B, i.e.
\(P_{f}(A) = P_{i}(A|B)\).

The last Bayesian idea is confirmation. According to \emph{SEP}, if for
an evidence (with proposition B) we have \(P_{i}(A|B) > P_{i}(A)\), then
the evidence confirms A. In other words, if after gaining the evidence
and conditionalize on A we get \(P_{f}(A) > P_{i}(A)\), then the
evidence confirms A. Finally, I want to contrast a principle of
entailment that does hold true with Confirmation of Entailments which
White argues to not hold. According to \emph{SEP}, whenever A entails B,
i.e. \(A \Rightarrow B\), then B confirms A and also \(P(A) \leq P(B)\).
On the other hand, Confirmation of Entailments says that "if E confirms
H which entails H', then E confirms H'" holds (532). So, if
\(H \Rightarrow H^{'}\) and E confirms H, White does not think it is
always true that E confirms H'.

Now, I present White's objection with the scenario set up earlier.
Throughout the objection degrees of justification of a belief loosely
corresponds the degree of confidence in the belief. White assumes that
getting more justification for a proposition is incompatible with the
decrease in credence for the proposition. The first part of White's
objection shows that by this assumption, a result Dogmatism would want
cannot be derived. White observes since R entails I and F entails I,
\(P(I|R) = P(I|F) = 1\)\footnote{Strictly speaking, the probability
  cannot be 1, but only approximates closely to 1. As this distinction
  does not influence the inequality relations, I represent situations
  like this as equal to 1 for simplicity.}. So, \(P(I|R) > P(I)\) and
\(P(I|F) > P(I)\). By Bayes theorem, \(P(R|I) > P(R)\) as
\(P(I|R) = \frac{P(I)P(R|I)}{P(R)} > P(I)\), and similarly
\(P(F|I) > P(F)\). As \(P(F) + P(\neg F) = 1\) and
\(P(F|I) + P(\neg F|I) = 1\), \(P(\neg F|I) < P(\neg F)\). So far, the
probability relations hold true synchronically and is independent of the
experience E. Then, White assumes that experience E gives full
justification for I. The partial belief should be conditionalized such
that \(P_{f}(R) = P_{i}(R|I) > P_{i}(R)\) and
\(P_{f}(\neg F) = P_{i}(\neg F|I) < P_{i}(\neg F)\). So, proposition I
raises the agent's confidence in R while lower the confidence in
\(\neg F\). Dogmatism would want \(\neg F\) to be more justified by
experience E. However, by the assumption above, more justification in
\(\neg F\) cannot be compatible with lowered credence in \(\neg F\). The
steps in the first part of the argument can be summarized as the
following:
\begin{enumerate}
	\item \(P(I|R) = P(I|F) = 1\)

	\item \(P(I|R) > P(I)\) and \(P(I|F) > P(I)\)

	\item \(P(R|I) > P(R)\) and \(P(F|I) > P(F)\)

	\item \(P(\neg F|I) < P(\neg F)\)

	\item \(P_{f}(R) > P_{i}(R)\) and \(P_{f}(\neg F) < P_{i}(\neg F)\)
\end{enumerate}

By the principle of entailment shown above, since R entails \(\neg F\),
\(P(R|I) < P(\neg F|I)\). White claims that \(P(R|I)\) is inversely
proportional to \(P(I|F)P(F)\). Since \(P(I|F) = 1\) by (1), \(P(R|I)\)
is dependent on \(P(F)\) i.e. inversely proportional to \(P(F)\).
\(P(R|I)\) is also dependent on \(P(\neg F)\), i.e. proportional to
\(P(\neg F)\). Formally, \(P(R|I) < P(\neg F)\) can be derived from
\(P(R|I) < P(\neg F|I)\) and (4). So after experience E gives full
justification for I, conditionalization gives that
\(P_{f}(R) < P_{f}(\neg F)\). Using (5), we get
\(P_{f}(R) < P_{i}(\neg F)\). Therefore, justification for \(\neg F\) is
antecedent to justification for R. Adding these steps to the formulas:
\begin{enumerate}
	\item \(P(R|I) < P(\neg F|I)\)

	\item \(P(R|I) < P(\neg F)\)

	\item \(P_{f}(R) < P_{f}(\neg F)\)

	\item \(P_{f}(R) < P_{i}(\neg F)\)
\end{enumerate}
So, the second part of White's objection shows that the credence in R
after the experience will be capped by the credence in
$\neg F$ before the experience. In other words, in order for R
to attain full justification after the experience E, $\neg F$
has to be already in full justification before the experience (534).
White argues that if antecedent justification for $\neg F$ is
a necessary condition for the justification of R, then the justification
for R has to be a mediated rather than immediate as what Dogmatism
claims it to be.

\section{Miller's response to White}

The main argument from Brian Miller is that Bayesianism does not impose
requirements on how experience should be incorporated into the formal
system of credences in partial beliefs. In other words, Bayesian model
does not prescribe how confidence one should be in terms of forming
their belief upon certain experience. After presenting Miller's
argument, I assess whether it successfully refutes White's objection,
and then I will discuss the plausibility of Miller's account in
connecting Bayesianism and Dogmatism.

Miller's argument targets the equation
\(P_{f}(\neg F) = P_{i}(\neg F|I)\), which is needed to move from (4) to
(5). Miller argues that \(P_{f}(\neg F) = P_{i}(\neg F|I)\) can be
interpreted either as the Bayesian conditionalization upon having
\(P(I) = 1\), or the update one should have upon having the experience
E. If the statement is interpreted in the first way, then the result
only says about the capping effect of $\neg F$ to belief I,
rather than a problem for gaining confidence upon the experience E. In
other words, the second objection instead is that it is a necessary
condition to assign higher confidence to the falsehood of skeptical
scenario F for ones later rational confidence in R upon gaining the
introspective belief I. This result is not a problem for dogmatism as
dogmatism is a theory about beliefs upon experience rather than beliefs
upon introspection. If the statement is to be interpreted the second
way, then there is an assumption that is made for the argument but is
not from Bayesianism. Miller considers revision on the credence function
other than conditionalization as "exogenous" revision, as it is a change
in credence not by Bayesian principles (Miller 8). Dogmatism has to be
incorporated into the model as an exogenous revision. Such exogenous
revision is only constrained by logically invalid claim that renders
incoherence in credence function. For example, for mutually exclusive
propositions such as A and \(\neg A\), since they are disjoint,
\(P(A \vee \neg A) = P(A) + P(\neg A)\). Therefore exogenous update on A
and \(\neg A\) have to sum up to 1. Since coherence is the only
constraint, one needs to argue that the rational confidence change upon
experience E should just be represented as exogenous revision that sets
\(P(I) = 1\) only. Without such argument, the Bayesian argument cannot
go through directly.

Miller then shows that there is an alternative exogenous revision that
leads to a different result. If one represents Dogmatism in Bayesian model as saying that
having an experience E gives exogenous change \(P(I) = 1\) as well as
\(P(R) = 1\), then the second interpretation above fails. If one sets
\(P_{f}(R) = 1\) exogenously, then since \(P(F|R) = 0\),
\(P_{f}(F) = 0\). Then as \(P(F) + P(\neg F) = 1\),
\(P_{f}(\neg F) = 1\). Therefore,
\(P_{f}(\neg F) \neq P_{i}(\neg F|I) < P_{i}(\neg F)\). So one cannot
move from (4) to (5). If (5) does not hold, then (9) does not hold.
Similarly, even given (8), (9) does not follow as its derivation
requires \(P_{f}(\neg F) = P_{i}(\neg F|I) < P_{i}(\neg F)\)\footnote{Note
  that the inequalities that I used can be inequality with equality. But
  equality does not always holds and here I just showed that the
  equality doesn't always holds. So using strictly smaller than would
  not be a problem and it will be consistent with other philosophers'
  notation in this paper.}. Since Bayesianism cannot rationally require
that the agent to set \(P_{f}(I) = 1\) rather than \(P_{f}(R) = 1\), and
this alternative leads to the opposite result, the second interpretation
fails. As the first interpretation is invalid as a counterargument to
Dogmatism, the Bayesian objection fails.

Now, with Miller's objection laid out, I want discuss its success over
White's argument. White considered the objection along the line Miller
proposed and he thinks that the interpretation in the first way is
sufficient to pose problem for dogmatism, as introspective belief I is a
natural consequence from having an experience E by people reflecting on
their experience. Therefore, White may argue that granted reflection on
experience is not necessary in all cases, the objection at least showed
that for a majority of cases Dogmatism faces a problem. As for the
problem of modeling on introspective belief rather than experience,
White suggests that if we have that the rational response upon
experience E and the introspective belief I is decrease in confidence in
R, then we should rationally decrease the confidence in R when we only
have experience E but without the introspective belief I. White's
intuition behind this is that introspection is performed by the choice
of the agent and it is a rather contingent choice. So, the existence of
introspection should not make a difference in the epistemic result.
However, I think Miller is right in his analysis. Miller points out that
for the first interpretation, the key problem is that it is an attack
that is irrelevant with respect to experience. Even though an experience
is usually followed by introspection of the experience, White's
objection in the first interpretation does not present it as a
consequence or even related to experience. As the first interpretation
assumes that the update is purely formal by conditionalization, it is
difficult for White to apply the conclusion back to Dogmatism as a
consequence of the view. As for the intuition for the contingency of
introspection, Miller did not address this problem, but I think he may
argue that if we still take the first interpretation, then the result
created as a result of the contingent choice. If the introspection is
contingent, then the formal result of the objection is also contingent
as it depends on introspection.

Miller does not attempt to bridge Bayesianism with Dogmatism. Miller
restricts his objection to a negative account to the Bayesian objection,
rather than a positive account for representing Dogmatism in Bayesian
model. Miller seems to suggest that since the justification for R
according to Dogmatism is a defeasible justification, the credence
revision of R cannot be fully modeled by Bayesianism. Miller denies that
his objection eliminate the possibility that priori epistemic state
poses constrain on the epistemic attitude after the perceptual
experience. However, Miller thinks Bayesian model cannot show whether
credence in priori defeater F should be changed upon revision in R or
the credence of R should be constrained by the priori defeater. In
particular, the model does not show whether we should update on R and I,
thus limiting F, or update on only I and let F limit R. I think Miller's
account is plausible, and I want to argue its plausibility by some
insights from Pryor. Firstly, as noted before, there is a difference
between prima facie justification and all things considered
justification. Pryor notes that Dogmatism concerns with prima facie
justification, while Bayesianism works with all things considered
justification (Pryor 16). The problem with setting the credence in R
directly to almost 1 upon perceptual experience is that it lefts out the
possibility of background beliefs or priori defeaters to undermine or
override the prima facie justification. When an agent has experience as
if R and low credence in F, then increasing the credence on R should be
a rational exogenous revision. However, if the agent has the same
experience but high credence in F, then exogenous revision only on the
introspective belief I is the more rational move. So, acquiring
immediate prima facie justification for perception should not directly
correspond to the probability function going up, or updated in general.
Therefore, Miller's account in the gap between Bayesianism and Dogmatism
is plausible.

\section{Moretti's response to White}

Luca Moretti proposes a similar problem for the Bayesian objection as
Miller on White's using the introspective belief I for modeling the
objection. Moretti's main argument against the Bayesian objection rests
on the idea that introspective belief I overrules the experience E.
Moretti argues that the conclusion that White arrives in the objection
holds as a later stage result after the prima facie justification
provided by Dogmatism. I will focus on the latter part and evaluates its
plausibility.

Moretti first shows that Bayesian objection cannot directly hit
Dogmatism, just as Miller's first interpretation. In the Bayesian model,
White replaces the proposition R with the introspective belief I and
also replaces an experience, or an epistemic state, with a belief. So,
the model is deficient at the beginning and require justification on how
its result models deficiency in Dogmatism precisely.

Moretti then shows that the indirect Bayesian objection fails. Moretti
argues against the idea that experience and introspective belief have
the same evidential force for justification. If they do not have the
same evidential force, then the indirect argument cannot go through.
Firstly, Dogmatists would consider the evidential force of experience to
be strong while not as so strong for introspective belief. Secondly,
since R and I have different logical relation with F, where I is the
entailed by F but R is logically incompatible with F, they have
different evidential consequences for F. Since F entails I, I confirms F
by the principle of entailment laid out in White's section. So learning
I should increase the credence in F and thus decrease the credence in
$\neg F$. However as R is incompatible with F, experience as
if R (experience E) gives prime facie justification for R, which should
weaken but not strengthen for F. Rather, it should strengthen
justification for $\neg F$ because R entails
$\neg F$. This is consistent with White as White grants
Justification Closure, which says that "if S is justified in believing
P, and can tell that P entails Q, then other things being equal, S is
justified in believing Q" (White 528). Therefore, by closure, the
justification for $\neg F$ should be more justified as well.
By White's own assumption, increase in justification is incompatible
with decrease in credence. So, the result considering R is incompatible
with the result considering I. So directly, White's objection has no
force because of if one replaces introspective belief I with R, the
result will be different.

While Moretti agrees with Miller that updating on R will halt the force
of White's indirect objection (Miller's second interpretation), he does
not think such move is the appropriate move to take. He thinks that if
one updates on R, then one has to \emph{believe} R to the highest
possible degree\footnote{Moretti permits the highest possible degree to
  not be 1.}. However, if one believes R to the highest degree, then
he/she cannot rationally doubt P. As the justification for R upon the
experience E is prima facie, such belief in R should be able to be
rationally doubted. Moretti thinks that this problem arises because one
cannot equivocate \emph{believing} R to the highest degree with
\emph{experiencing} as if R.

In the last response to White, Moretti argues that the conclusion of
White's indirect objection is consistent with Dogmatism. Moretti thinks
that the problem lies in how an agent's justification for a proposition
should be determined when both experience and introspective belief I are
present. Moretti argues that the strength of justification for
proposition R is determined by introspective belief I when introspection
is present, and thus the experience E's contribution to the
justification of the proposition R becomes irrelevant. In other words,
even though E and I have different evidential force, when they are both
present, I determines the change in justification for proposition R. So,
when introspective belief I is present, the all things considered
justification for the proposition is the same as the justification for
the proposition based on the introspective belief I. Note that this
claim of overruling is different from White's claim that experience with
introspection has the same result as experience alone.

Moretti justifies this claim through a thought experiment. I modified
the experiment to use the abbreviations I set up before. Consider that
there are two white balls and two red balls in a bag. One of the white
balls is painted to appear to the agent as a red ball. The rest are left
unpainted. The belief F above can be modified as "I am deceived to think
there is a red ball when there is a white ball". The belief
$\neg F$ can be modified as "I am not deceived to think there
is a red ball when there is a white ball". The agent has his/her eyes
shut and pull out a ball from the bag. Given that the agent has been
told about the balls' colors in the bag and about the painting, the
credence for the ball corresponds to the actual probability.
\(P(R) = \frac{1}{2}\) since there are two red balls out of four.
\(P(I) = \frac{3}{4}\) since there are three balls that look red out of
four. \(P(\neg F) = \frac{3}{4}\) since there is only one deception
ball, i.e. \(P(F) = \frac{1}{4}\).
\(P(I|R) = \frac{P(I \land R)}{P(R)} = \frac{\frac{1}{2}}{\frac{1}{2}} = 1\)\footnote{P(I\^{}R)
  is actually 3/8 (by 3/4 * 1/2), but Moretti rounds it up to 1/2.} and
similarly we have
\(P(I|\neg F) = \frac{P(I \land \neg F)}{P(\neg F)} = \frac{\frac{1}{2}}{\frac{3}{4}} = \frac{2}{3}\)
\footnote{P(I\^{}$\neg F$) is actually 9/16 (by 3/4 * 3/4),
  but Moretti rounds it up to 1/2.}. Therefore, according to Bayes
Theorem, \(P(R|I) = P(\neg F|I) = \frac{2}{3}\). By conditionalization
of I, we get \(P_{f}(R) = P_{i}(R|I) = \frac{2}{3}\) and
\(P_{f}(\neg F) = P_{i}(\neg F|I) = \frac{2}{3}\). Note that
\(P(\neg F) = \frac{3}{4}\) is larger than \(P_{f}(\neg F)\), so the
credence for $\neg F$ should drop. So far, the analysis
corresponds with introspective belief without experience. Now, with the
agent's eyes open and the agent has the experience E. The agent then
gains prima facie justification for fully believing R by Dogmatism
account. Since R entails $\neg F$, by Justification Closure,
the agent gains prima facie justification for fully believing
$\neg F$. Now, after the experience, the agent has
introspection about his/her experience and comes to believe I. Since the
agent already \emph{knows} the credences on propositions by I, all
things considered justifications for R and $\neg F$ lowers to
their credence by I alone. Aside from the thought experiment, Moretti
justifies the intuition for overdetermination of introspective belief by
appealing to daily practice. Since perceptual justifications are often
only attributed to individuals unable to reflect on their experience,
such justifications are overruled by introspective justifications for
individuals who are able to have introspection on their experience
(Moretti 21).

Given the overdetermination of introspective belief, Moretti argues that
White's argument fails to show that the sole experience as if R
(experience E) lowers justification for $\neg F$. After
gaining experience, without introspective belief, the agent gains prima
facie justification for fully believing R and then by Justification
Closure fully believing $\neg F$ at \(t_{1}\). After having
introspection, the agent comes to have introspective belief I. Then,
with overdetermination, the all things considered justification for
$\neg F$ agrees with its justification by I at \(t_{2}\).
$\neg F$'s justification by I at \(t_{2}\) agrees with the
analysis of credence for $\neg F$ by White. The lowering of
credence corresponds to the lowering of all things considered
justification. So the justification for $\neg F$ at \(t_{2}\)
is lower than justification for $\neg F$ before introspection
at \(t_{1}\). Therefore, White's objection is not targeted at the prima
facie justification that is at the center of Dogmatism, but rather the
all things considered justification. Prima facie justification can still
strengthen the justification for $\neg F$ before
introspection, so the argument fails at showing the inconsistency
between Dogmatism and Bayesianism.

I think Moretti's first two arguments work well against White, although
the reply to the indirect objection may be subject to criticism on its
loose use of conversion between credence and justification, such as
White's Confirmation of Entailment. If White is right about Confirmation
of Entailment being wrong, then the reply needs more careful formulation
using Justification Closure and converts the justification degree to
credences. I do not think Moretti's third argument work as well. I agree
with Moretti in distinguishing between the state of justification for
sole experience and for experience with introspection. I think this may
correspond to the difference in prima facie justification and all things
considered justification. However, I do not think the state of
justification for introspection alone overdetermines the state for
experience with introspection. Firstly, the thought experiment Moretti
used has a major flaw. Before the agent has the experience of perceiving
the ball, the agent has already known the existence of visual illusion
in the set up, i.e. having very strong defeating evidence. This is
evident in the credence of propositions Moretti gives before the visual
experience. Indeed, with the possession of defeating evidence, all
things considered justification will be lower than prima facie
justification. However, this is not in general the case, or rarely the
case where the agent posses strong defeating evidence for the perceptual
experience. Secondly, I don't think the overdetermination is intuitively
true, because introspective belief is treated as a defeating evidence or
overriding evidence that replaces the prima facie justification in all
cases. I don't think mere introspection would be a defeating evidence in
all cases. The ball appearing to me is red does not seem to be an
undermining evidence to the claim that there is a red ball in front of
me.

\section{Conclusion}

In this paper, I presents formulations of Dogmatism and Bayesian
objection to Dogmatism. I analyzes two replies to the Bayesian objection
in defense of Dogmatism. Both replies are true in pointing out the
modeling imperfection of Dogmatism in the Bayesian objection. They both
point out that the Bayesian modeling is not directly about experience
and respectively targets the exogenous revision to the model and
evidential differences that fail to be modeled. The difficulty of the
topic lies in the positive account that establishes the connection
between Bayesianism and Dogmatism. I supported Miller's claim in the gap
between the two accounts and criticized Morettie's argument to combine
Dogmatism and Bayesianism into a diachronic model that includes prima
facie justification and all things considered justification.
\clearpage
\section*{Appendix}

Here is a "conversion table" that may be helpful in checking between the
symbol system used in different papers.

\subsection*{\normalsize White's formulation} 

\begin{tabular}{rp{8cm}}
	H1: 	& it appears to me that this is a hand = I\\

	H2: 	& this is a hand = R\\

	H3: 	& this is not a fake-hand (I am not a handless brain in a vat) =
$\neg F$\\

	not-H3:	& this is a fake hand = F
\end{tabular}

\subsection*{\normalsize Miller's formulation}

\begin{tabular}{rp{8cm}}
	BIV: 	& I am a handless brain in a vat having experiences (as of my hands) = F\\

	e: 	& I am having an experience as if I have hands = I\\

	h: 	& I have hands = R
\end{tabular}

\subsection*{\normalsize Moretti's formulation}
\begin{tabular}{rp{8cm}}
	E: 	& it appears to me that this is a hand = I \\

	P: 	& this is a hand = R\\

	SH: 	& this is a fake-hand = F \\

	R: 	& the wall is red = R\\
	B: 	& it appears to me that the wall is red = I\\
	SH*: & the wall is white but looks red because it is illuminated by a
hidden red light =F
\end{tabular}
\clearpage
\section*{References}
{
\small
\begin{itemize}[label={},itemindent=-2em,leftmargin=2em]	
	\item Miller, Brian T. ``How to Be a Bayesian Dogmatist.'' \emph{Australasian
Journal of Philosophy}, vol. 94, no. 4, 2016, pp. 766--780.

	\item Moretti, Luca. ``In Defence of Dogmatism.''~\emph{Philosophical
Studies}, vol. 172, no. 1, 2014, pp. 261--282.,
doi:10.1007/s11098-014-0293-0.

	\item Pryor, James. "The skeptic and the dogmatist." \emph{Noûs}, 2000, 34
(4):517--549.

	\item Pryor, James. ``Problems for Credulism.'' \emph{Seemings and
Justification}, 2013, pp. 89--132.

	\item Talbott, William, "Bayesian Epistemology", \emph{The Stanford
Encyclopedia of Philosophy} (Winter 2016 Edition), Edward N.
Zalta (ed.)

	\item White, Roger. ``Problems for Dogmatism.'' \emph{Philosophical Studies},
vol. 131, no. 3, 2006, pp. 525-557., 
\end{itemize}
}